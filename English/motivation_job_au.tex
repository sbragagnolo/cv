% -*-coding: utf-8 -*-
%%%%%%%%%%%%%%%%%%%%%%%%%%%%%%%%%%%%%%%%%
% Plain Cover Letter
% LaTeX Template
% Version 1.0 (28/5/13)
%
% This template has been downloaded from:
% http://www.LaTeXTemplates.com
%
% Original author:
% Rensselaer Polytechnic Institute 
% http://www.rpi.edu/dept/arc/training/latex/resumes/
%
% License:
% CC BY-NC-SA 3.0 (http://creativecommons.org/licenses/by-nc-sa/3.0/)
%
%%%%%%%%%%%%%%%%%%%%%%%%%%%%%e%%%%%%%%%%%%

%----------------------------------------------------------------------------------------
%	PACKAGES AND OTHER DOCUMENT CONFIGURATIONS
%----------------------------------------------------------------------------------------

\documentclass[11pt]{letter} % Default font size of the document, change to 10pt to fit more text
\usepackage[utf8]{inputenc}
\usepackage{newcent} % Default font is the New Century Schoolbook PostScript font 
%\usepackage{helvet} % Uncomment this (while commenting the above line) to use the Helvetica font

% Margins
\topmargin=-1in % Moves the top of the document 1 inch above the default
\textheight=8.5in % Total height of the text on the page before text goes on to the next page, this can be increased in a longer letter
\oddsidemargin=-10pt % Position of the left margin, can be negative or positive if you want more or less room
\textwidth=6.5in % Total width of the text, increase this if the left margin was decreased and vice-versa

%\let\raggedleft\raggedright % Pushes the date (at the top) to the left, comment this line to have the date on the right

\begin{document}

%----------------------------------------------------------------------------------------
%	ADDRESSEE SECTION
%----------------------------------------------------------------------------------------

\begin{letter}{} 
%\name{Santiago Bragagnolo}
\date{}%29 Juin 2015}
%----------------------------------------------------------------------------------------
%	YOUR NAME & ADDRESS SECTION
%----------------------------------------------------------------------------------------

%\begin{flushright}
%\raggedleft{
%Lettre de Motivacion\\
%Santiago Bragagnolo % Your name
%\vspace{20pt} \hrule height 1pt % If you would like a horizontal line separating the name from the address, uncomment the line to the left of this text
%14 Rue Denis du Peage \\ Lille, Nord pas de Calais 59800 \\ (+33) 06-52-70-66-13 \\santiagobragagnolo@gmail.com% Your address and phone number
%}
%\end{flushright} 

\signature{Santiago Bragagnolo} % Your name for the signature at the bottom

%----------------------------------------------------------------------------------------
%	LETTER CONTENT SECTION
%----------------------------------------------------------------------------------------
%29 Juin 2015

Subject: Presentation

\opening{To whom it may concern} 
 
% Primer parrafo: laburo actual
 
  I am a technology transference and research engineer of the InriaTech team at INRIA, Lille, since april 2015. 
  I am responsable for the maturation and transfer of the technological outline of the work or research,  as well as implementing research prototypes.  
  Actually, I am related with many different frames projects of many different research teams as RMOD, Non-A, FUN and Spirals. 
  With these teams, I work in different subjects namely: robotics, smart contracts on blockchain databases, real time, IOT networking, programming languages conception and implementation, software profiling in terms of energy consumption.
  In order to lead this projects to a good destination, i make usage of different software engineering techniques and it associated technologies. 
  To arrive to deliver good quality products I use version control systems (VCS), Continuous integration (CI), Continuous delivery (CD), founded on a solid base of unitary test and integration tests, outline of the proper usage of techniques as Test and behaviour driven development  (TDD/BDD). 
  In order to meet the requirements of these pieces of software, I use different development methods. Based on the needed maturity and the available time I may use pure iterative or agile methods. 
  Finally, to arrive to a good communication in between the different actors related to the development, I use mainly Domain Driven Development techniques for the naming and the general software conception, aiming to develop a common language of communication. And I use pair programming, for sharing know-how and to unify working methodologies. 
  
  
% Douai
Previously I worked on L'Ecole des mines de Douai, as an esearch engineer for 18 months (fixed time contract). In this position, I played a similar rol to the current one, working with the robotic research team named "CAR". I was contracted for the development of a framework for the implementation of strategic level behaviours for robots. Including also the implementation of a supermarket robot guide.

% Industria
Besides my four year experience within the French research, I worked also as software engineer in the industry for twelve years in Argentina and Spain, on many different subjects.
My career allowed me to use different \textbf{Programming languages}: Python, Java, C\#, C/C++, PHP, Javascript, Scala, Groovy, SQL, PL/SQL, T-SQL; many \textbf{technologies}: relational and non-relational databases (Oracle, SQL Server, MySql, Postgre SQL, MongoDB, DB4J, Hadoop, Hive), social networks interaction (Facebook, Twitter), operative systems  (Linux, Windows), mobile platforms (Android Smartphones and tablets), geographic information systems; many  \textbf{methodologies}: Extreme programming, Scrum, Kanban, TDD, BDD, DDD; many 
\textbf{paradigms}: object oriented, component oriented, functional, aspect oriented; and also many \textbf{architectures}: client-server (with both, heavy and light clients), distributed processing, multilayer (traditional three layers and hexagonal), services (rest and soap) and micro services. 

% Experiencia academica
Parallelly, my academic experience in Argentina is also rich. It consists in \textbf{study} of almost four years of the career of Software engineering in a recognised institution. I also did my high school in a professional institution under the same subject, with the title of computing technician, in one of the most important technics institutions of Argentina. 
\textbf{experimentation}: I had being part of many experimentation projects on different subjects, namely: lambda calculus, type inference, monads, parallel and concurrent processing;  Finally, important for my life as a mean of personal realisation \textbf{teaching}: I worked as ad-honorem assistant on subjects related with software implementation and conception, programming paradigms, and application architecture. Currently I am working on having an accreditation for my experience for having a title from a french university, being represented by INRIA it self.

% strengths
During all my career I proofed to be a quick learner, being that one of my biggest strengths, allowing me to be quite versatile.  With this heterogeneous experience in diametrically different domains, I have the capacity to extrapolate concepts from different domains and being able to offer  many non-traditional points of views and solutions. Finally, having developed my self in different cultures, environments and in different roles, as developer, database administrator, researcher and team leader, I gained a strong understanding of many roles related to the software development's life cycle, mixed with the mastering of three languages (English, Spanish and French) makes me a good communicator and team unifier. 

%Besides my general history and strengths, I have many years experience working with research teams related with research done together with Thales, on subjects of modularisation and re-engineering.


 \textbf{As interesting information, I am in Melbourne in between 1 July and 14 July 2017. If my profile is interesting for you, we may be able to arrange a face to face meeting on these days. }
 


\closing{Cordially,}



%----------------------------------------------------------------------------------------

\end{letter}

\end{document}