%%%%%%%%%%%%%%%%%%%%%%%%%%%%%%%%%%%%%%%%%
% Medium Length Professional CV
% LaTeX Template
% Version 2.0 (8/5/13)
%
% This template has been downloaded from:
% http://www.LaTeXTemplates.com
%
% Original author:
% Trey Hunner (http://www.treyhunner.com/)
%
% Important note:
% This template requires the resume.cls file to be in the same directory as the
% .tex file. The resume.cls file provides the resume style used for structuring the
% document.
%
%%%%%%%%%%%%%%%%%%%%%%%%%%%%%%%%%%%%%%%%%

%----------------------------------------------------------------------------------------
%	PACKAGES AND OTHER DOCUMENT CONFIGURATIONS
%----------------------------------------------------------------------------------------

\documentclass{resume} % Use the custom resume.cls style

\usepackage[left=0.75in,top=0.6in,right=0.75in,bottom=0.6in]{geometry} % Document margins
\usepackage{hyperref}



\name{Santiago Bragagnolo} % Your name
	\address{28, Rue Clovis Hugues \\ Lille, 59800 } % Your address
	\address{(+33)~$\cdot$~7~$\cdot$~83~$\cdot$~14~$\cdot$~45~$\cdot$~35 \\ santiagobragagnolo@gmail.com} % Your phone number and

\begin{document}



\begin{rSection}{Biographic data}

\begin{tabular}{ @{} >{\bfseries}l @{\hspace{6ex}} l }
	Complete name & Santiago Pablo Bragagnolo  \\
	Birth date & 16 november 1982  \\
	Nationalities & Argentinian, Italian  \\
	Visiting Melbourne &   1 July - 14 July 2017. Face to face interview is possible on these days. \\
	EU eligibility & YES  \\
\end{tabular}

\end{rSection}


\begin{rSection}{Contact}

\begin{tabular}{ @{} >{\bfseries}l @{\hspace{6ex}} l }
	Skype & santiago.bragagnolo  \\
	Linkedin & linkedin.com/in/santiagobragagnolo/  \\
	Portfolio & santiagobragagnolo.wordpress.com  \\
	Blog & knowledgeconvergence.wordpress.com  \\
	Github & github.com/sbragagnolo \\
	SmalltalkHub & smalltalkhub.com/\#!/~sbragagnolo \\
\end{tabular}

\end{rSection}


%----------------------------------------------------------------------------------------
%	TECHNICAL STRENGTHS SECTION
%----------------------------------------------------------------------------------------


\begin{rSection}{Experience Brief}

\begin{tabular}{ @{} >{\bfseries}l @{\hspace{6ex}} l }

Software Industry experience & 11 years \\
Research experience & 4 years \\
Effective engineering study & 3.5 years \\
Teaching programming on university & 5 years \\

\end{tabular}
\end{rSection}
\begin{rSection}{Technical Strengths}

\begin{tabular}{ @{} >{\bfseries}l @{\hspace{6ex}} l }
Paradigms & Object Oriented, Functional \\
Languages & Pharo, Scala, Java, C/C++, Javascript, Python  \\
Application Middleware & Tomcat, JBoss, Google App Engine \\
MVC Frameworks &  Seaside, SpringMVC, Grails, Play2 \\
Web Client Frameworks & Backbone.js, Underscore.js, JQuery \\
SQL Databases & MySQL, PostgreSQL, Oracle, SQLServer  \\
Blockchain Databases & Ethereum \\
NO-SQL Databases & MongoDB, BigTable, Db4o  \\
Tools & Metacello, Monticello, Make, Maven, SVN, GIT, Eclipse \\
Processing Architectures & Multithreading, Multiprocesses, Real time \\
Software Architectures & REST, Service,  Client \& Server, Data Processing (ETL), Actors\\
Quality/Testing & SUnit, JUnit, Jenkins \\
IDEs & Pharo, Squeak, Dolphin, Eclipse, Netbeans, Visual studio \\
Operative system & Ubuntu/Debian Linux, CentOS Linux / Redhat enterprise \\
\end{tabular}

\end{rSection}

\begin{rSection}{Methodology Strengths}

\begin{tabular}{ @{} >{\bfseries}l @{\hspace{6ex}} l }
	Working methodologies & Agile, KanBan, XP, Pair programming  \\
	Software design & Domain Driven Design  \\
	Quality assurance & Continuous Integration, Continuous Delivery \\
	Development  & Test driven development, Behaviour driven development, Lean development \\
\end{tabular}

\end{rSection}

\begin{rSection}{Yet unusual experiences}

\begin{tabular}{ @{} >{\bfseries}l @{\hspace{6ex}} l } 
	Blockchain - Ethereum & INRIA \\ 
	Indoor robotics  & INRIA + Ecole de mins de douai. 
\end{tabular}

\end{rSection}

\begin{rSection}{Languages}

\begin{tabular}{ @{} >{\bfseries}l @{\hspace{6ex}} l }
	Spanish & Native \\
	English & Written and spoken fluent competence \\
	French & Written and spoken fluent competence \\
	Chinese & spoken basic low competence \\
\end{tabular}

\end{rSection}



%----------------------------------------------------------------------------------------
%	WORK EXPERIENCE SECTION
%----------------------------------------------------------------------------------------


\begin{rSection}{Currently working on}

	\begin{rSubsection}{INRIA}{April 2015 - currently }{Position Definition - Technological transfer Engineering}
		\item Research and applied research engineering position, meant to develop bilateral contracts in between INRIA and enterprises.
		\item To mature and adapt the technologies developed by different INRIA's research teams for satisfying the needs of the industry. 
		\item To participate in the presentation the INRIA's technological offer.
		\item https://www.inria.fr/en/centre/lille/news/inriatech-initiative-in-lille-inria-s-support-for-research-and-innovation-award
	\end{rSubsection}
	
	\begin{rSubsection}{INRIA SPIRALS - Technological transfer project}{June 2017 - Currently }{Energy consumption analysis as a service}
		\item 
		\item Power API enhancement (http://powerapi.org/) 
		\item Development of a service platform for serving Power API. 
		\item Project developed on Scala + AKKA and Java. 
	\end{rSubsection}
	
	\begin{rSubsection}{INRIA RMOD - Technological transfer project}{October 2016 - Currently }{Blockchain's Smart contract analysis }
		\item 
		\item Research analysis over the blockchain's smart contracts subject.
		\item Prototype development for exploring research possibilities.
		\item Development on Solidity, Javascript and Pharo.
		\item Conception and implementation of a Solidity's language interpreter for the generation of inspection tools.
		\item Conception and implementation of a Ethereum virtual machine simulator with advanced debugging utilities. 
		\item Conception and implementation of a Ethereum virtual machine Byte code decompiler. 
		\item Conception and implementation of graphical navigation tools for Ethereum. 
		\item Conception and implementation of query language for Ethereum. 
		\item Writing of scientific articles
		\item The ongoing work has been presented on PharoDays 2017 conference  (http://pharo.org/2017PharoDays) as "Navigating the blockchain"
		\item https://www.slideshare.net/sbragagnolo1/a-living-programming-environment-for-a-living-blockchain
	\end{rSubsection}
	
	\begin{rSubsection}{INRIA Non-A - Technological transfer project}{December 2015 - Mars 2016 }{Sephyr - Intelligent table }
		\item 
		\item Conception and development of a voice commanded robotised table
		\item Modification and adaption of local and global planning algorithms for the specificity of the domain 
		\item State machine based protocol analyser for voice command recognition 
		\item Voice command module driver.
		\item Robot's physical model definition 
		\item Developed with Python, C++ and Pharo, Designed to run in a ROS environment. 
		\item https://www.inria.fr/en/centre/lille/news/zephyr-the-intelligent-voice-activated-overbed-table
	\end{rSubsection}
	
	
	\begin{rSubsection}{Inria RMOD -Maturation of technology}{June 2016 - Currently }{Scale}
		\item 
		\item Library for scripting and REPL for Pharo
		\item Developed with Pharo and BASH. 
		\item Tested and automatically builded by travis
		\item https ://github.com/guillep/Scale		
	\end{rSubsection}
	
	\begin{rSubsection}{INRIA Non-A -Maturation of technology}{April 2015 - Currently }{MakROS}
		\item Data-transfert based component oriented framework for robotic behaviours ROS compatible. 
		\item Using for development Pharo. Conceived for fast behaviour prototyping.
		\item Includes multiple default components, giving full support to the basic ROS library needs, as TF/TF2, in a component fashion.
		\item You can check it out from: 
		\item https://github.com/sbragagnolo/Makros
	\end{rSubsection}
	
	\begin{rSubsection}{INRIA Non-A}{April 2015 - Currently }{MetaDDS/SimpleDDS/ROSDDS}
			\item Skeleton and canon implementation of a Data Delivery Service (publisher/subscriber) framework and it full ROS and Pharo smalltalk specific implementation. Based on ROS standard and OMG-DDS standard.
			\item Using for development Pharo
			\item Extra features for robustness as node reconnection and master restart recovery. 
			\item Tested and automatically builded by travis
			\item You can checkit out from: 
			\item https://github.com/sbragagnolo/MetaDDS
			\item https://github.com/sbragagnolo/SimpleDDS
			\item https://github.com/sbragagnolo/ROSDDS
	\end{rSubsection}

	\begin{rSubsection}{Inria RMOD - Maturation of technology}{April 2015 - Currently }{TaskIT}
		\item 	
		\item Maintenance and development of TaskIT framework. 
		\item TaskIT is an object oriented framework for concurrent and parallel processing. Simple, debuggable and powerful, mean to meet real time requirements in a Green thread based environment.
		\item  https ://github.com/sbragagnolo/TaskIT/main
	\end{rSubsection}
	

	\begin{rSubsection}{Inria RMOD  Maturation of technology }{June 2016 - December 2016 }{Android Pharo VM}
		\item		
		\item Conception and development of the adaption of the CogVM virtual machine for running on Android 
		\item Developed on Java Android, C++ and Pharo.
		\item Building automation.
		\item Tested and automatically builded by travis
 		\item https ://github.com/sbragagnolo/pharo-vm	
	\end{rSubsection}


	\begin{rSubsection}{Inria Non-A - Maturation of technology}{April 2015 - June 2016 }{Trajectory algorithms}
		\item 
		\item Implementation of trajectory algorithms based on adaptive control methods
		\item Developed in C++ . Prepared to run on ROS environments. 
		\item Tested and builded automatically on a private Jenkins
	\end{rSubsection}




\end{rSection}


\begin{rSection}{Industry and Research Experience}

\begin{rSubsection}{Ericsson}{May 2014 - September 2014 }{ Senior Java and C++ consultant developer }{Malaga, Spain}
\item ERA (Ericsson ran analyzer) Project. C++ and Java based high-performance client
\item TPS (Trace processor server) Project. Java high-performance server for trace processing (system that faces problems of teras of data) with some IronPython for processing text files, with part of processing over hadoop and hive.
\item ODG (OSS Data gateway) Project. Project that analyze data from OSS servers. Also Java high-performance server. 
\item All of these project are highly Multithreaded and based on distributed processing. 
\item The data containers of these project are based on Hadoop File system. 
\item Most of the data processed is reduced and stored into the client mapping data from several servers with workflow processing. (Based in ETL pattern)
\item TDD and Continuous delivery evangalizer.
\item Working under Agile with Scrum methodology
\end{rSubsection}


\begin{rSubsection}{Ecole des mines de Douai}{September 2012 - February 2014}{Software Research Engineer, Robotics}{Douai, Nord-Pas-de-Calais}
\item I am in charge of the development of the project RoboShop (http://car.mines-douai.fr/RoboShop), a robotic system for aiding persons to navigate into unknown spaces. I do organize the work also of two persons more, an intership and a post-doc.  
\item My responsibilities are planning, engineering, development and technologic report writing for the projects RoboShop and PhaROS (http://car.mines-douai.fr/PhaROS). 
\item PhaROS is a framework for developing robot solutions for Pharo Smalltalk on the robotic middleware ROS. It implements a real time distributed architecture.
\item The project RoboShop was presented in Picom and Vad Conext 2013 (Some content about http://car.mines-douai.fr/2013/11/roboshop-demo-16oct2013/) 
\item The project PhaROS has been presented by me in FOSDEM 2014, (https://fosdem.org/2014/) as PhaROS Towards Live Environments in Robotics in the Smalltalk devroom
\item https://www.slideshare.net/sbragagnolo1/fosdem-1
\item It was also presented by Dr Noury Bouraqadi on different conferences
\item https://www.slideshare.net/nourybouraqadi/robo-shop
\item https://www.slideshare.net/nourybouraqadi/roboshop-helper-robot-in-a-shopping-mall
\item From this position i also contribute with the project TaskIT (http://smalltalkhub.com/\#!/~sbragagnolo/TaskIT), which is a project for managing parallelism and concurrence in a Pharo smalltalk environment.
\item The used technologies for the robot side are: Pharo, ROS, Python and C++. For the graphical interface: Pharo, Seaside, Bootstrap, Javascript, html+css3
\end{rSubsection}


%------------------------------------------------

\begin{rSubsection}{Fanwards (http://www.fanwards.com/)}{November 2011 - August 2012}{Software engineer, gamification}{Ciudad de Buenos Aires, Argentine}
\item I am in charge of design and implementation of the web application Fanwards, front and backend. working with the CTO.
\item Using for front end heavy weight clients based on javascript technologies such as Backbone.js, jquery, underscore.js , mustache and for the view HTML5 and CSS3.
\item Using for the backend a Google App Engine (GAE) Server with Scala \& Java, Objectify for the mapping between objects and google's BigTable, Spring MVC and RestFul frameworks for routing and dispatching of exposed and scheduled behaviours.Finally twitter4j and facebookRest for interacting with social networks.  
\item From this position i have developed: a full intelligent single-page client based on javascript,  a small functional library for javascript implementing some of the common haskell features (partial application, curryfication and function compositions), really useful abstractions for AJAX request processing, also developed a social network crawler for analysing users comments for branding (with heuristics to analyse the meaning of each comment). 
\item From this position i also participate not just in software design, planning and architecture but also in the gamification process of the application.
\item During all the development of the application we used Scala BDD and TDD techniques with great success.
\item Crossing data for brand reports with Map reduce. 
\item You can watch the most important part of the implementation in the following example \url{https://www.facebook.com/pages/Melee-Island-Inc/252598398140724?id=252598398140724&sk=app_232320516837452}
\end{rSubsection}



\begin{rSubsection}{Aufiero Informatica}{March 2011 - November 2011}{Software architect, designer \& developer}{Ciudad de Buenos Aires, Argentine}
\item This company is a small software factory and AVG antivirus reseller for latino america. I worked two times in this enterprise, look down in previous experiences to look my progress. In this case i was also in charge of a 5 person team.
\item Reseller/Partner management system based on Groovy on Grails technology and communicating to software legacy done in PHP.
\item Mail campaign web system for internal usage (For AVG campaigns). Done in Groovy and Grails / Jasper reports
\item Email send system multi-engine, auto-deployable with load balance and mail tracking. Done in Groovy and Java, using Apache Email. 
\item Single sign on system for our different platforms - Done in Groove and grails.
\end{rSubsection}


\begin{rSubsection}{Buscouniversidad .com}{January 2011 -  March - 2011}{Developer \& system designer}{Ciudad de Buenos Aires, Argentine}
\item In this work i'am in charge of designing of Data base structure, SQL queries. In the design of the directory system (such as OLX, Craiglists, etc, but specific for universities)  based on PHP with Zend framework, sphinx and Javascript with JQuery.  And in the development of processing tools
\item From this position i had developed mail processing based on patterns for recognising rejects and angry people (python); an easy code generator for Zend framework; the administration application of the site. 
\end{rSubsection}

\begin{rSubsection}{Aufiero Informatica}{March 2007 - December 2010}{Software designer \& developer - Project manager }{Ciudad de Buenos Aires, Argentine}
\item In this company i had being in several projects, always as software designer and developer, and in the last year also as architect. 
\item The followings are the project i did and my role
\item Accountant management system - Designer and developer - Java, Flex 3.1 JBoss 
\item Editorial integral management system (stock, accounting, finances, sells, shopping, etc) - Maintainer, developer and DBA - visual basic 6.0 / sql server / Crystal reports. Implementation of ETL pattern for making up a small datawarehouse for sales analysis.
\item Intranet tool for client management Pharo/Squeak - PostgreSQL - Seaside. 
\item Classified ads management system - Maintainer, developer  - Net Framework 3.0 C\# Nhibernate Windows form
\item Billing online system - Software designer, developer - PHP, CakePHP, Javascript jquery. In this project we were contracted for offshoring by Ixole, a company from Barcelona. I worked in this case also as face of our team, dealing about deadlines and requirements. This project lasted one year.
\item Electronic Invoice system (Based on the local taxes system)  -  Software designer, developer - Java, Groovy and Grails / jasper reports.
\end{rSubsection}


\begin{rSubsection}{MSA}{2006}{DBA Oracle 9i \& Postgres SQL}{Ciudad de Buenos Aires, Argentine}
\item In this company i had in charge the administration of three productive databases related with ticket system (kind of ticketek but with less stress) 
\item My tasks were Data base monitoring, Backup, SQL Security, Database and query tuning for the productive systems (Oracle 9i) and making up configurations for eventual projects (Usually Postgres SQL)
\item Parallely i had some small responsibilities in eventual projects 
\end{rSubsection}


\begin{rSubsection}{Research for decision}{2003 - 2005}{Developer}{Ciudad de Buenos Aires, Argentine}
\item In this company my main responsibility was the development of polls in eole/saxophone (language and poll system).
\item Also i was in charge of maintaining servers, machines and network.
\end{rSubsection}

\begin{rSubsection}{Freelance}{2002 - 2009}{Developer, Designer, Architect, DBA \& Client management}{Ciudad de Buenos Aires, Argentine}
\item This are the projects i have developed as independent, several of them still in usage, and i maintained almost them for one to two years. 
\item  November 2002 to July 2003 - Nutritional preparations management system for a Parenteral Laboratory (UNANUT) -  Visual Basic \& Microsoft Access ( Supported until 2005) 
\item  June 2003 to August 2003 - Stock management system for a box factory  - Visual Basic \& Microsoft Access ( Supported until 2004) 
\item  July 2009 to March 2010 - Certification and courses management system SFAP -  PHP (http://www.facpce.org.ar/)  (Supported until 2011)
\end{rSubsection}



\end{rSection}


%----------------------------------------------------------------------------------------
%	OPEN SOURCE SECTION
%----------------------------------------------------------------------------------------
%------------------------------------------------
\begin{rSection}{Experience in open source - community projects}


\begin{rSubsection}{On my own}{October 2014 - Currently }{MakROS}
\item Data-transfert based component oriented framework for robotic behaviours ROS compatible. 
\item Using for development Pharo
\item Includes multiple default components, giving full support to the basic ROS library needs, as TF/TF2, in a component fashion.
\item You can checkit out from: 
%	\begin{itemize}
		\item https://github.com/sbragagnolo/Makros
%	\end{itemize}
\end{rSubsection}

\begin{rSubsection}{On my own}{October 2014 - Currently }{MetaDDS/SimpleDDS/ROSDDS}
\item Skeleton and canon implementation of a Data Delivery Service (publisher/subscriber) framework and it full ROS and Pharo smalltalk specific implementation. Based on ROS standard and OMG-DDS standard.
\item Using for development Pharo
\item Extra features for robustness as node reconnection and master restart recovery. 
\item Tested and automatically builded by travis
\item You can checkit out from: 
%	\begin{itemize}
		\item https://github.com/sbragagnolo/MetaDDS
		\item https://github.com/sbragagnolo/SimpleDDS
		\item https://github.com/sbragagnolo/ROSDDS
%	\end{itemize}
\end{rSubsection}

\begin{rSubsection}{On my own}{Feburary 2014 - Currently }{TaskIT}
\item Maintaining and enhancing TaskIT with Guillermo Polito
\item TaskIT had born and grow during the development of PhaROS during my job at Ecole des mines, and it was mean to be object oriented, easy to use, and with powerful, well tested, simple and reliable processing abstractions, in order to serve in a real-time processing environment. 
\item As well it provides some experimental functionalities for process understanding as the plan definition. 
\item We are currently working on Task-IT 2.0, which will have a great new suit of concept for process management
\item You can checkit out from https://github.com/sbragagnolo/TaskIt
\end{rSubsection}



\begin{rSubsection}{On my own}{Mars - May 2014}{Making work Scala + Play + MongoDB}{Malaga}
	\item Putting together some existant technology for a project on my own
	\item Fully developed with Scala and Play. It has some java dependancies (Jackson for JSon marshalling)
	\item https://github.com/sbragagnolo/mongodb
\end{rSubsection}

\begin{rSubsection}{On my own}{June - August 2014}{Social secure plugin for Play running on top of MongoDB}{Malaga}
	\item Social secure backend implementation for running on mongoDB
	\item Fully developed with Scala and Play. 
	\item https://github.com/sbragagnolo/SocialSecurePlayMongo
\end{rSubsection}


\begin{rSubsection}{Google summer of code}{May 2012 - September 2012}{Type inference on dynamic languages}{Ciudad de Buenos Aires, Argentine}
\item I am in charge of design, planning, research, engineering, responding to my mentor Nicolas Passerini (npasserini@gmail.com). 
\item The proposal is registered http://gsoc2012.esug.org/projects/type-inference 
\item From this project i have implemented a concrete type inference system for Pharo Smalltalk 1.4, a graph of methods to be executed as response of the analysis of a given expression,  and an object oriented logger, which is now being used in several Pharo projects. I also took the work of blogging all the work progress in a blog. This project was presented in the conference ESUG 2012 - Gent - (http://www.esug.org/wiki/pier/Conferences/2012) 
\item Site - http://concretetypeinference.blogspot.fr/. 
\item Concrete type inferencer and the call graph analyser (Kwisatz Haderach) - http://ss3.gemstone.com/ss/ConcreteTypeInference.html
\item Paule le poulpe; Object oriented Logger - http://smalltalkhub.com/mc/sbragagnolo/PLP/main
\end{rSubsection}

 
\begin{rSubsection}{ESUG financed Project}{ May 2011 - September 2011}{Developer in DBXTalk}{Ciudad de Buenos Aires, Argentine}
\item In this project i am in charge of design and implementation of the scaffolding for DBXTalk (DBXTools) and the bridging with Glorp framework.
\item DBXTalk (http://dbxtalk.smallworks.com.ar/) is a bridge that gives support for the  database systems to mainstream Squeak and Pharo smalltalk. 
\item This project with the porting of Glorp to Pharo (by Guillermo Polito) were presented in ESUG 2011 - Edimburgh (http://www.esug.org/wiki/pier/Conferences/2011)
\end{rSubsection}

\begin{rSubsection}{FACPCE financed Project - Used for SFAP }{ 2009 }{Developer in Cornucopia framework}{Ciudad de Buenos Aires, Argentine}	
\item PHP Cornucopia. Is an already obsolete full stack framework. It provides:
	\begin{itemize}
		\item Collections
		\item Simple ORM configurable by metadata 
		\item Simple dependency injector
		\item HTML reification for composing and HTML generation
		\item Javascript generation by metadata (Requirements, etc)
		\item Request reification as object
		\item Session reification as object
	\end{itemize}
\item Check it out from: https://github.com/sbragagnolo/cornucopia
\end{rSubsection}


\begin{rSubsection}{On my own for an university topic}{ 2007 }{Developer in C Objects framework}{Ciudad de Buenos Aires, Argentine}
\item C Objects. Is a framework to provide several object oriented programming features to C. It provides:
	\begin{itemize}
		\item Collections (Dictionary, List) 
		\item Strings
		\item Server, with node objects for managing requests
		\item Automatons for simple language interpretation 
		\item Threads reification 
		\item Mutex and Conditional reifications
		\item Error management (managed by signals, giving the change to register error handlers and to raise errors) 
		\item Memory management.
	\end{itemize}
\item Checkit out from: https://github.com/sbragagnolo/c-objetos

\end{rSubsection}


\end{rSection}

%----------------------------------------------------------------------------------------
%	TEACHING SECTION
%----------------------------------------------------------------------------------------

\begin{rSection}{Teaching experience}


\begin{rSubsection}{Universidad Tecnologica Nacional (UTN)}{March 2007 - July 2012}{Adhonorem teaching assistant at Advanced programming techniques}{Ciudad de Buenos Aires, Argentine}
\item Teaching the next concepts, techniques and tools
\begin{itemize}
	\item Object oriented programming
	\begin{itemize}
		\item Patterns
		\item Methodologies TDD/BDD, DDD, Agile
		\item Re-factors
		\item Meta-programming
	\end{itemize}
	\item Basic architectures
	\item Technologies
	\begin{itemize}
		\item Maven 
		\item IDEs - Eclipse, Idea, Netbeans
		\item JUnit, ScalaTest
		\item SVN \& GIT
	\end{itemize}
	\item Dynamic languages 
	\begin{itemize}
		\item  Scala
		\item  Python
		\item  Smalltalk
		\item  Self
	\end{itemize}
	\item Modern applied concepts
	\begin{itemize}
		\item Traits and Mixins
		\item  Lambdas / anonymous functions
	\end{itemize}
\end{itemize}
\end{rSubsection}

\begin{rSubsection}{Universidad Tecnologica Nacional (UTN)}{March 2007 - December 2011}{Adhonorem teaching assistant at Programming paradigms}{Ciudad de Buenos Aires, Argentine}
\item Teaching the next concepts, techniques and tools
\begin{itemize}
	\item Object oriented paradigm
	\begin{itemize}
		\item Pharo Smalltalk
	\end{itemize}
	\item Functional oriented paradigm
	\begin{itemize}
		\item GHC
		\item WinHugs
	\end{itemize}
	\item Logical oriented paradigm 
	\begin{itemize}
		\item  Swi prolog
	\end{itemize}
\end{itemize}
\end{rSubsection}


\begin{rSubsection}{Fundacion Proydesa}{March 2006 - November 2007}{Oracle DBA Instructor}{Ciudad de Buenos Aires, Argentine}
\item In this foundation i worked as instructor of three of the four basic modules of Oracle for Database administration
\item SQL (Module 1)
\item Engine Architecture (Module 2)
\item Tuning (Module 4)
\item I also participate as instructor in the instruction of new instructors for modules 1 and 2.
\end{rSubsection}


\end{rSection}



%----------------------------------------------------------------------------------------
%	EDUCATION SECTION
%----------------------------------------------------------------------------------------

\begin{rSection}{Education}


{\bf Universidad Tecnologica Nacional (UTN) - Ciudad de Buenos Aires - Argentine} \hfill {\em January 2004 - Aborted on 2011; having papers to proof 7 semesters of career of 10 semesters } \\ 
Software Engineer  \\

{\bf Ing. Otto Krause - Ciudad de Buenos Aires - Argentina} \hfill {\em December 2001} \\ 
Technic in computation \\
\end{rSection}

%----------------------------------------------------------------------------------------
%	EXAMPLE SECTION
%----------------------------------------------------------------------------------------

%\begin{rSection}{Section Name}

%Section content\ldots

%\end{rSection}

%----------------------------------------------------------------------------------------

\end{document}
