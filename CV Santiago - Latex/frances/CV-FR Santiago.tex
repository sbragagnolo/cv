%%%%%%%%%%%%%%%%%%%%%%%%%%%%%%%%%%%%%%%%%
% Medium Length Professional CV
% LaTeX Template
% Version 2.0 (8/5/13)
%
% This template has been downloaded from:
% http://www.LaTeXTemplates.com
%
% Original author:
% Trey Hunner (http://www.treyhunner.com/)
%
% Important note:
% This template requires the resume.cls file to be in the same directory as the
% .tex file. The resume.cls file provides the resume style used for structuring the
% document.
%
%%%%%%%%%%%%%%%%%%%%%%%%%%%%%%%%%%%%%%%%%

%----------------------------------------------------------------------------------------
%	PACKAGES AND OTHER DOCUMENT CONFIGURATIONS
%----------------------------------------------------------------------------------------

\documentclass{resume} % Use the custom resume.cls style

\usepackage[left=0.75in,top=0.6in,right=0.75in,bottom=0.6in]{geometry} % Document margins
\usepackage{hyperref}


\name{Santiago Bragagnolo} % Your name
\address{345 Rue d'Arras \\ Douai, Nord-pas-de-calais 59500} % Your address
\address{(+33)~$\cdot$~06~$\cdot$~95~$\cdot$~55~$\cdot$~37~$\cdot$~39 \\ santiagobragagnolo@gmail.com} % Your phone number and email

\begin{document}

%----------------------------------------------------------------------------------------
%	EDUCATION SECTION
%----------------------------------------------------------------------------------------

\begin{rSection}{Education}

{\bf Universidad Tecnologica Nacional (UTN) - Ciudad de Buenos Aires - Argentine} \hfill {\em Janvier 2004 - 2011 (Inconcluse)} \\ 
Software Engineer - J'ai course  3 first years and part of the forth  (Of 5 years of career) \\

{\bf Ing. Otto Krause - Ciudad de Buenos Aires - Argentina} \hfill {\em December 2001} \\ 
Technic in computation \\
\end{rSection}

%----------------------------------------------------------------------------------------
%	WORK EXPERIENCE SECTION
%----------------------------------------------------------------------------------------

\begin{rSection}{Experience}

\begin{rSubsection}{Ecole des mines de Douai}{September 2012 - Present}{Software Engineer, Robotics}{Douai, Nord-Pas-de-Calais}
\item I am in charge of the development of the project RoboShop (http://car.mines-douai.fr/RoboShop), a robotic system for aiding persons to navigate into unknown spaces. 
\item My responsibilities are planning, engineering, development and technologic report writing for the projects RoboShop and PhaROS (http://car.mines-douai.fr/PhaROS). 
\item PhaROS is a framework for developing robot solutions for Pharo Smalltalk on the robotic middleware ROS. 
\item The project RoboShop was presented in Picom and Vad Conext 2013 (Some content about http://car.mines-douai.fr/2013/11/roboshop-demo-16oct2013/) 
\item The project PhaROS is going to be presented in FOSDEM 2014, (https://fosdem.org/2014/) as PhaROS Towards Live Environments in Robotics in the Smalltalk devroom.
\item From this position i also contribute with the project TaskIT (http://smalltalkhub.com/\#!/~sbragagnolo/TaskIT), which is a project for managing parallelism and concurrence in a Pharo smalltalk environment, and it will be soon integrated as an enterprise solution for the language.
\item The used technologies for the robot side are: ROS, Pharo, C++, Python. For the graphical interface: Pharo, Seaside, Javascript+backbone.js + Underscore.js, html+css3
\item For the framework project we started to write a cook-book and the related documentation, the first one in latex format, and the second one in markdown format.
\end{rSubsection}


%------------------------------------------------

\begin{rSubsection}{Fanwards (http://www.fanwards.com/)}{November 2011 - August 2012}{Software engineer, gamification}{Ciudad de Buenos Aires, Argentine}
\item I am in charge of design and implementation of the web application Fanwards, front and backend. working with the CTO.
\item Using for front end heavy weight clients based on javascript technologies such as Backbone.js, jquery, underscore.js , mustache and for the view HTML5 and CSS3.
\item Using for the backend a Google App Engine (GAE) Server with Scala \& Java, Objectify for the mapping between objects and google's BigTable, Spring MVC and RestFul frameworks for routing and dispatching of exposed and scheduled behaviours.Finally twitter4j and facebookRest for interacting with social networks.  
\item From this position i have developed: a full intelligent single-page client based on javascript,  a small functional library for javascript implementing some of the common haskell features (partial application, curryfication and function compositions), really useful abstractions for AJAX request processing, also developed a social network crawler for analysing users comments for branding (with heuristics to analyse the meaning of each comment). 
\item From this position i also participate not just in software design, planning and architecture but also in the gamification process of the application.
\item During all the development of the application we used Scala BDD and TDD techniques with great success. 
\item You can watch the most important part of the implementation in the following example \url{https://www.facebook.com/pages/Melee-Island-Inc/252598398140724?id=252598398140724&sk=app_232320516837452}
\end{rSubsection}

%------------------------------------------------

\begin{rSubsection}{Google summer of code}{May 2012 - September 2012}{Type inference on dynamic languages}{Ciudad de Buenos Aires, Argentine}
\item I am in charge of design, planning, research, engineering, responding to my mentor Nicolas Passerini (npasserini@gmail.com). 
\item The proposal is registered http://gsoc2012.esug.org/projects/type-inference 
\item From this project i have implemented a concrete type inference system for Pharo Smalltalk 1.4, a graph of methods to be executed as response of the analysis of a given expression,  and an object oriented logger, which is now being used in several Pharo projects. I also took the work of blogging all the work progress in a blog. This project was presented in the conference ESUG 2012 - Gent - (http://www.esug.org/wiki/pier/Conferences/2012) 
\item Site - http://concretetypeinference.blogspot.fr/. 
\item Concrete type inferencer and the call graph analyser (Kwisatz Haderach) - http://ss3.gemstone.com/ss/ConcreteTypeInference.html
\item Object Logger - http://smalltalkhub.com/mc/sbragagnolo/PLP/main
\end{rSubsection}

 
\begin{rSubsection}{ESUG financed Project}{ May 2011 - September 2011}{Developer in DBXTalk}{Ciudad de Buenos Aires, Argentine}
\item In this project i am in charge of design and implementation of the scaffolding for DBXTalk (DBXTools) and the bridging with Glorp framework.
\item DBXTalk (http://dbxtalk.smallworks.com.ar/) is a bridge that gives support for the  database systems to mainstream Squeak and Pharo smalltalk. 
\item This project with the porting of Glorp to Pharo (by Guillermo Polito) were presented in ESUG 2011 - Edimburgh (http://www.esug.org/wiki/pier/Conferences/2011)
\end{rSubsection}


\begin{rSubsection}{Aufiero Informatica}{March 2011 - November 2011}{Software architect, designer \& developer}{Ciudad de Buenos Aires, Argentine}
\item This company is a small software factory and AVG antivirus reseller for latino america. I worked two times in this enterprise, look down in previous experiences to look my progress
\item Reseller /Partner management system based on Groovy on Grails technology and communicating to software legacy done in PHP.
\item Mail campaign web system for internal usage (For AVG campaigns). Done in Groovy and Grails / Jasper reports
\item Email send system multi-engine, auto-deployable with load balance and mail tracking. Done in Groovy and Java, using Apache Email. 
\item Single sign on system for our different platforms - Done in Groove and grails.
\end{rSubsection}


\begin{rSubsection}{Buscouniversidad .com}{January 2011 -  March - 2011}{Developer \& system designer}{Ciudad de Buenos Aires, Argentine}
\item In this work i'am in charge of designing of Data base structure, SQL queries. In the design of the directory system (such as OLX, Craiglists, etc, but specific for universities)  based on PHP with Zend framework, sphinx and Javascript with JQuery.  And in the development of processing tools
\item From this position i had developed mail processing based on patterns for recognising rejects and angry people (python); an easy code generator for Zend framework; the administration application of the site. 
\end{rSubsection}


\begin{rSubsection}{Universidad Tecnologica Nacional (UTN)}{March 2007 - July 2012}{Adhonorem teaching assistant at Advanced programming techniques}{Ciudad de Buenos Aires, Argentine}
\item Teaching the next concepts, techniques and tools
\begin{itemize}
	\item Object oriented programming
	\begin{itemize}
		\item Patterns
		\item Methodologies TDD/BDD, DDD, Agile, Scrum
		\item Re-factors
		\item Meta-programming
	\end{itemize}
	\item Basic architectures
	\item Technologies
	\begin{itemize}
		\item Maven 
		\item IDEs (Eclipse)
		\item JUnit
		\item SVN \& GIT
	\end{itemize}
	\item Dynamic languages 
	\begin{itemize}
		\item  Python
		\item  Smalltalk
		\item  Self
	\end{itemize}
	\item Modern applied concepts
	\begin{itemize}
		\item Traits and Mixins
		\item  Lambdas / anonymous functions
	\end{itemize}
\end{itemize}
\end{rSubsection}

\begin{rSubsection}{Universidad Tecnologica Nacional (UTN)}{March 2007 - December 2011}{Adhonorem teaching assistant at Programming paradigms}{Ciudad de Buenos Aires, Argentine}
\item Teaching the next concepts, techniques and tools
\begin{itemize}
	\item Object oriented paradigm
	\begin{itemize}
		\item Pharo Smalltalk
	\end{itemize}
	\item Functional oriented paradigm
	\begin{itemize}
		\item GHC
		\item WinHugs
	\end{itemize}
	\item Logical oriented paradigm 
	\begin{itemize}
		\item  Swi prolog
	\end{itemize}
\end{itemize}
\end{rSubsection}






\begin{rSubsection}{Aufiero Informatica}{March 2007 - December 2010}{Software designer \& developer}{Ciudad de Buenos Aires, Argentine}
\item In this company i had being in several projects, always as software designer and developer, and in the last year also as architect. 
\item The followings are the project i did and my role
\item Accountant management system - Designer and developer - Java, Flex 3.1 JBoss
\item Editorial integral management system (stock, accounting, finances, sells, shopping, etc) - Maintainer, developer and DBA - visual basic 6.0 / sql server / Crystal reports 
\item Classified ads management system - Maintainer, developer  - Net Framework 3.0 C\# Nhibernate Windows form
\item Billing online system - Software designer, developer - PHP, CakePHP, Javascript jquery
\item Electronic Invoice system (Based on the local taxes system)  -  Software designer, developer - Java, Groovy and Grails / jasper reports.
\end{rSubsection}


\begin{rSubsection}{MSA}{May 2006 - October 2006}{DBA Oracle 9i \& Postgres SQL}{Ciudad de Buenos Aires, Argentine}
\item In this company i had in charge the administration of three productive databases related with ticket system (kind of ticketek but with less stress) 
\item My tasks were Data base monitoring, Backup, SQL Security, Database and query tuning for the productive systems (Oracle 9i) and making up configurations for eventual projects (Usually Postgres SQL)
\item Parallely i had some small responsibilities in eventual projects 
\end{rSubsection}


\begin{rSubsection}{Fundacion Proydesa}{March 2006 - November 2007}{Oracle DBA Instructor}{Ciudad de Buenos Aires, Argentine}
\item In this foundation i worked as instructor of three of the four basic modules of Oracle for Database administration
\item SQL (Module 1)
\item Engine Architecture (Module 2)
\item Tuning (Module 4)
\item I also participate as instructor in the instruction of new instructors for modules 1 and 2.
\end{rSubsection}

\begin{rSubsection}{Research for decision}{January 2005 - November 2005}{Developer}{Ciudad de Buenos Aires, Argentine}
\item In this company my main responsibility was the development of polls in eole/saxophone (language and poll system)
\item Also i was in charge of maintaining servers, machines and network.
\end{rSubsection}

\begin{rSubsection}{Independent work}{2002 - 2009}{Developer, Designer, Architect, DBA \& Client management}{Ciudad de Buenos Aires, Argentine}
\item This are the projects i have developed as independent, several of them still in usage, and i maintained almost them for one to two years. 
\item  November 2002 to July 2003 - Nutritional preparations management system for a Parenteral Laboratory (UNANUT) -  Visual Basic \& Microsoft Access ( Supported until 2005) 
\item  June 2003 to August 2003 - Stock management system for a box factory  - Visual Basic \& Microsoft Access ( Supported until 2004) 
\item  July 2009 to March 2010 - Certification and courses management system SFAP -  PHP (http://www.facpce.org.ar/)  (Supported until 2011)
\end{rSubsection}




\end{rSection}

%----------------------------------------------------------------------------------------
%	TECHNICAL STRENGTHS SECTION
%----------------------------------------------------------------------------------------


\begin{rSection}{Technical Strengths}

\begin{tabular}{ @{} >{\bfseries}l @{\hspace{6ex}} l }
Operative system & Ubuntu, Fedora, Windows XP, Windows 7, Mac OS Mountain Lion \\
Computer Languages & Java, Scala, Javascript, C\#, Pharo, C, C++, Python  \\
Application Middleware & Tomcat, JBoss \\
Robotics Middleware & ROS \\
ORM Frameworks &  Hibernate, nHibernate, Objectify \\
DI Frameworks &  Spring  \\
MVC Frameworks &  SpringMVC, Grails \\
Client Frameworks & Backbone.js, Underscore.js, \\
Robotic Frameworks &  PhaROS, ROSCpp, ROSPy \\
Databases & MySQL, PostgreSQL, Oracle, SQLServer  \\
Virtualization & Virtualbox \\
Tools & SVN, GIT, Eclipse \\
Writing & LaTex, Markdown \\
\end{tabular}

\end{rSection}

%----------------------------------------------------------------------------------------
%	EXAMPLE SECTION
%----------------------------------------------------------------------------------------

%\begin{rSection}{Section Name}

%Section content\ldots

%\end{rSection}

%----------------------------------------------------------------------------------------

\end{document}
