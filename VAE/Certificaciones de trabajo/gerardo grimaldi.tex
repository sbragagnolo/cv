% -*-coding: utf-8 -*-
%%%%%%%%%%%%%%%%%%%%%%%%%%%%%%%%%%%%%%%%%
% Plain Cover Letter
% LaTeX Template
% Version 1.0 (28/5/13)
%
% This template has been downloaded from:
% http://www.LaTeXTemplates.com
%
% Original author:
% Rensselaer Polytechnic Institute 
% http://www.rpi.edu/dept/arc/training/latex/resumes/
%
% License:
% CC BY-NC-SA 3.0 (http://creativecommons.org/licenses/by-nc-sa/3.0/)
%
%%%%%%%%%%%%%%%%%%%%%%%%%%%%%e%%%%%%%%%%%%

%----------------------------------------------------------------------------------------
%	PACKAGES AND OTHER DOCUMENT CONFIGURATIONS
%----------------------------------------------------------------------------------------

\documentclass[11pt]{letter} % Default font size of the document, change to 10pt to fit more text
\usepackage[utf8]{inputenc}
\usepackage{newcent} % Default font is the New Century Schoolbook PostScript font 
%\usepackage{helvet} % Uncomment this (while commenting the above line) to use the Helvetica font

% Margins
\topmargin=-1in % Moves the top of the document 1 inch above the default
\textheight=8.5in % Total height of the text on the page before text goes on to the next page, this can be increased in a longer letter
\oddsidemargin=-10pt % Position of the left margin, can be negative or positive if you want more or less room
\textwidth=6.5in % Total width of the text, increase this if the left margin was decreased and vice-versa

%\let\raggedleft\raggedright % Pushes the date (at the top) to the left, comment this line to have the date on the right

\begin{document}

%----------------------------------------------------------------------------------------
%	ADDRESSEE SECTION
%----------------------------------------------------------------------------------------

\begin{letter}{} 
%\name{Santiago Bragagnolo}
\date{}%29 Juin 2015}
%----------------------------------------------------------------------------------------
%	YOUR NAME & ADDRESS SECTION
%----------------------------------------------------------------------------------------

%\begin{flushright}
%\raggedleft{
%Lettre de Motivacion\\
%Santiago Bragagnolo % Your name
%\vspace{20pt} \hrule height 1pt % If you would like a horizontal line separating the name from the address, uncomment the line to the left of this text
%14 Rue Denis du Peage \\ Lille, Nord pas de Calais 59800 \\ (+33) 06-52-70-66-13 \\santiagobragagnolo@gmail.com% Your address and phone number
%}
%\end{flushright} 

\signature{Santiago Bragagnolo} % Your name for the signature at the bottom

%----------------------------------------------------------------------------------------
%	LETTER CONTENT SECTION
%----------------------------------------------------------------------------------------
%29 Juin 2015

Objet: Validation des experiences acquises : Master 2

\opening{Madame, Monsieur,} 
 
% Primer parrafo: laburo actual

Je suis à présent ingénieur-transfert technologique dans l'équipe InriaTech d'Inria Lille - Nord Europe. depuis avril 2015. Je suis responsable du maturation et transfert technologique des résultats de recherche a la industrie. 
Actuellement je suis lie au projets de investigation dans le cadre de différents equipes de recherche, comme RMOD, Non-A et FUN. 
Avec ces equipes je travaille sur different thématiques a savoir, robotique, contracts intelligents sur blockchain, temp reel, réseaux des objets connectés, implementation des langages de programmation.
Au but de diriger ces maturations de projets au bon port,  je saisis différents techniques de genie logiciel d'avant garde et ses technologies associes. 
Pour arriver a la bonne qualité de produit, j'utilise  Gestion de versions (VCS), integration continue(CI), livraison continue(CD), basée sur solid base de tests unitaires et tests d'integration lie aux methods de développement piloté par les tests et par comportement (TDD/BDD).
Pour arriver a la bonne cible, j'utilise différents methods de développement, selon la maturité attendu et le temps disponible, j'utilise des techniques purement itératif ou agiles. 
Finalement pour arriver a la bonne entente du domain et a la bonne communication avec les différents acteurs lie au développement, je utilise principalement la conception pilotée par le domaine (DDD) et la programmation en binôme pour aller le plus loin possible en partager la comprehension du domain et façon de faire. 


% Douai
Précédemment, j'ai travaille a la Ecole des mines de Douai comme ingénieur de recherche pendant 18 mois (contrat CDD). Au cette position j'ai joué un rôle similaire à ce que j'ai aujourd'hui, en travaillent avec l'equipe de recherche robotique "CAR", sur la genie logiciel sur les sujet de développement des solutions pour le control stratégique des robots. 


% Industria
En plus de mon experience de presque quatre ans en genie logiciel lieu a la recherche en France, j'ai aussi travaillé comme ingénieur logiciel pendant douze ans sur la industrie du logiciel en Argentine et en Espagne, sur sujet aussi divers. 
Cet carrier m'ai laisse utiliser plusieurs \textbf{languages de programmation}: Java, C\#, C/C++, PHP, Javascript, Scala, Groovy, SQL, PL/SQL; 
plusieurs \textbf{technologies}: Base de donees relationnelles et non-relationnelles (Oracle, SQL Server, MySql, Postgre SQL, MongoDB, DB4J, Hadoop, Hive), connections au réseaux sociaux (Facebook, Twitter), systèmes d'exploitation (Linux, Windows), platforms portables (Smartphones et Tablets Android), systèmes d'information géographique; 
plusieurs \textbf{methodologies}: Extreme programming, Scrum, Kanban, TDD, BDD, DDD; 
plusieurs \textbf{paradigmes}: Orienté objets, orienté components, fonctionnel, aspects;  et aussi plusieurs \textbf{architectures}: client-serveur (client léger et client lourd), traitement distribué, multicouche (traditionnelle de trois couche, et hexagonale) , services rest / soap et micro services.


% Experiencia academica
En parallèle, mon experience académique en Argentine, c'est aussi riche et consiste en \textbf{etudes}: j'ai fais vingt et un cours (plus de trois ans) de la carrière de genie logiciel dans une institution reconnu. j'ai fais un bac professional sur le meme sujet (technicien des computations) dans un des institutions de techniciens plus important d'Argentine;   \textbf{experimentation}: j'ai participe dans plusieurs projets et ateliers d'experimentation de différentes sujet, comme lambda calcule, inference de types, monads, processus parallèle et concurrent;  et aussi important pour ma vie en tant de realisation personal, \textbf{l'enseignement}: j'ai travaillé comme assistant d'enseignement ad-honorem sur des cours lie au développement logiciel au tant d'implementation, conception, paradigmes et architecture.


% por que diploma
La raison principale de mon demande de VAE est la possibilité d'une continuation académique de ma carrière. Je suis vraiment intéressé pour avoir la possibilité de faire une thèse et de me développer comme professionnel, scientific et personne, et continuer a contribuer à la recherche et à la société française comme je fais il a des années, mais  d'une manier plus cadré, et avec un impact plus efficace . 


% Por que ISEN
%Concrètement, Je vous adresse ponctuellement a vous ISEN, pour plusieurs raisons. Les deux raisons plus remarquable sont, d'abord, vous trajectoire et nom, a mon avis, incontournables. Et deuxième le fait de que ma carrière et vous programme ont parallélisme indéniable.

% Por que lille 1
Concrètement, après d'avoir travaillé avec les professionnels chercheurs et ingénieurs lie a la Université de Lille 1. Soit comme partenariat inter-équipes de recherche à l'Ecole des Mines de Douai. Soit comme aujourd'hui, comme partie de la même famille en INRIA. j'ai eu la chance travailler, côte à côte, avec des excellents professionnelles de la taille du Jean Pierre Richard, Wilfrid Perruquetti, Stephane Ducasse, Anne Etien, et des professionnelles gradués de Lille 1 comme Guillermo Polito et Christophe Demarey, qui sont franchement inspirateurs. 

% Conclusion

Je suis persuadé que ce titre me permettra avoir une reconnaissance formelle de mes connaissances. Et, aussi, me permettra avoir un moyen de suivre avec ma carrière académique et pouvoir me faire une place pour former les suivants generations citoyens professionnelles de cette société que j'aime.

Je vous remercie de l'attention que vous portez à ma candidature au VAE. Je vous prie, Madame, Monsieur, d'agréer mes salutations distinguées.


\closing{Cordialement,}



%----------------------------------------------------------------------------------------

\end{letter}

\end{document}