

%%%%%%%%%%%%%%%%%%%%%%%%%%%%%%%%%%%%%%%%%
% Medium Length Professional CV
% LaTeX Template
% Version 2.0 (8/5/13)
%
% This template has been downloaded from :
%  http://www.LaTeXTemplates.com
%
% Original author :
% Trey Hunner ( http://www.treyhunner.com/)
%
% Important note :
% This template requires the resume.cls file to be in the same directory as the
% .tex file. The resume.cls file provides the resume style used for structuring the
% document.
%
%%%%%%%%%%%%%%%%%%%%%%%%%%%%%%%%%%%%%%%%%

%----------------------------------------------------------------------------------------
%	PACKAGES AND OTHER DOCUMENT CONFIGURATIONS
%----------------------------------------------------------------------------------------

\documentclass{resume} % Use the custom resume.cls style
\usepackage[utf8]{inputenc}
\usepackage[left=0.75in,top=0.6in,right=0.75in,bottom=0.6in]{geometry} % Document margins
\usepackage{hyperref}



\name{Santiago Bragagnolo} % Your name
	\address{28 rue Clovis Hugues \\ Lille, 59800 } % Your address
	\address{(+33)~$\cdot$~7~$\cdot$~83~$\cdot$~14~$\cdot$~45~$\cdot$~35 \\ santiagobragagnolo@gmail.com} % Your phone number and
\begin{document}



\begin{rSection}{Donn\'{e}es Biographiques}

\begin{tabular}{ @{} >{\bfseries}l @{\hspace{6ex}} l }
	Pronom et Nom & Santiago Pablo Bragagnolo  \\
	Date de naissance & 16 november 1982  \\
	Nationalit\'{e}s & Argentin, Italien  \\
\end{tabular}

\end{rSection}


\begin{rSection}{Contact}

\begin{tabular}{ @{} >{\bfseries}l @{\hspace{6ex}} l }
	Skype & santiago.bragagnolo  \\
	Linkedin & linkedin.com/in/santiagobragagnolo/  \\
	Portfolio & santiagobragagnolo.wordpress.com  \\
	Blog & knowledgeconvergence.wordpress.com  \\
	Github & github.com/sbragagnolo \\
\end{tabular}

\end{rSection}

%----------------------------------------------------------------------------------------
%	WORK EXPERIENCE SECTION
%----------------------------------------------------------------------------------------


\begin{rSection}{Equivalences avec le program du Master en informatique - Infrastructures applicatives et génie logiciel}


	% Semestre 1 
	\begin{rSubsection}{Algorithmes et complexité}{Semestre 1}{ Crédits 5 }
		\item Cours universitaire (952021) Algorithms et structures de donnes
		\item Cours universitaire (952030) Gestion de donnes
		\item 2 années d'experience avec l'équipe de recherche mathematic NON-A sur la robotique 
	\end{rSubsection}
	\begin{rSubsection}{Architecture conception des systèmes d'exploitation}{Semestre 1}{ Crédits 5 }
		\item 3 années d'experience comme DBA et administrateur de systèmes.
		\item 2 années d'experience comme enseignant 
	\end{rSubsection}
	\begin{rSubsection}{Architecture évoluée des ordinateurs}{Semestre 1}{ Crédits 5 }
		\item Cours universitaire (952022) Architecture de processeur 
	\end{rSubsection}
	\begin{rSubsection}{Projet encadré}{Semestre 1}{ Crédits 5 }
		\item 13 années d'experience dans plusieurs genre de projets IT.
	\end{rSubsection}
	\begin{rSubsection}{UE5 Anglais  DPP}{Semestre 1}{ Crédits 5 }
		\item 5 années d’experience en équipes multi culturels
	\end{rSubsection}	
	\begin{rSubsection}{Génie logiciel}{Semestre 1}{ Crédits 5 }
		\item Cours universitaire (952028) Conception du système 
	\end{rSubsection}
	% Semestre 2
	\begin{rSubsection}{Projet individuel}{Semestre 2}{ Crédits 5 }
		\item Conception et implementation de TaskIT. Util pour générer programmation concurrent et parallèle. 
		\item Conception et implementation de ROSDDS. Util pour connecter le langage de programmation Pharo au middleware ROS. 
		\item Conception et implementation de Mako. Framework de development de components.
		\item Conception et implementation de Makros. Framework de development de components robotics.
	\end{rSubsection}
	\begin{rSubsection}{Construction des applications réparties}{Semestre 2}{Crédits 5}
		\item Conception et implementation de systèmes robotics avec ROS
		\item Plusieurs années d'experience de conception et implementation de services rest et soap.
		\item 1 année d'experience en développement de sites internet sur la platform Google App Engine.
	\end{rSubsection}
	\begin{rSubsection}{Langages avancés pour les Bases de Données}{Semestre 2}{ Crédits 5 }
		\item 3 années de Oracle et PostgreSQL DBA , 
		\item 3 années d’experience T-SQL. 
		\item 7 années experience SQL/ Hibernate/ LinQ .net
		\item Experience en Conception et implementation de languages de requête. 
	\end{rSubsection}
	\begin{rSubsection}{Programmation parallèle et distribuée}{Semestre 2}{ Crédits 5 }
		\item Conception et implementation de client ROS pour Pharo
		\item Conception et implementation de TaskIT. Util pour générer programmation concurrent et parallèle. 
		\item 1 année d'experience sur des systèmes d'haut performance en Ericsson. 
	\end{rSubsection}
	\begin{rSubsection}{Algorithmes et Applications }{Semestre 2}{ Crédits 5 }
		\item Experience avec NON-A, développement des algorithms. 
		\item Type inference algorithms. 
	\end{rSubsection}
	\begin{rSubsection}{ Architecture avancée des systèmes d'exploitation}{Semestre 2}{ Crédits 5 }
		\item Cours universitaire (952027) Syst\`{e}mes d'exploitation 

	\end{rSubsection}
	\begin{rSubsection}{ Administration des bases de données }{Semestre 2}{ Crédits 5 }
		\item 3 années de Oracle et PostgreSQL DBA
	\end{rSubsection}

	% Semestre 3
	\begin{rSubsection}{Outils pour la programmation des logiciels}{Semestre 3}{ Crédits 6 }
		\item IDES: Visual studio 2009/2010, Visual basic 5, Pharo / Squeak,  Eclipse, Intellij, Atom
		\item Gestion de version: SVN, GIT
		\item Integration continuée: Travis, Jenkins
	\end{rSubsection}
	\begin{rSubsection}{Conception Agile des Logiciels}{Semestre 3}{ Crédits 6 }
		\item 5 années de travaille sur la méthodologie  Agile scrum
		\item 2 années de travaille sur a  méthodologie Kanban
	\end{rSubsection}
	\begin{rSubsection}{Intelligence des données et des Logiciels}{Semestre 3}{ Crédits 6 }
		\item 3 années d'experience sur reports ventes et stock (sur SQL-Server) 
		\item 2 années d'experience sur reports des enquêtes. (sur SPSS, excel et SQL-Server )  
		\item 3 annes de experience avec robotique appliquée. 
		\item 
	\end{rSubsection}
	\begin{rSubsection}{Initiation à l'innovation et à la recherche }{Semestre 3}{ Crédits 6 }
		\item 3 années de travaille avec des équipes de recherche
		\item Actuellement en écrivant un article scientific sur la blockchain 
	\end{rSubsection}
	\begin{rSubsection}{Infrastructures et frameworks internet}{Semestre 3}{ Crédits 6 }
		\item 10 années d’experience avec web apps
		\item Browser side frameworks: J-Query, Backbone Js, Moustache Js. 
		\item Server side frameworks: Spring grails, Spring mvc, CakePHP, Zend, Seaside, Play 1.0/2.0. 
	\end{rSubsection}
	
	% Semestre 4 
	\begin{rSubsection}{Communication et culture d'entreprise}{Semestre 4}{ Crédits 5 }
		\item 1 années de travaille comme chef d'équipe. 
	\end{rSubsection}
	\begin{rSubsection}{Anglais}{Semestre 4}{ Crédits 5 }
		\item 5 années d'experience dans des équipes multiculturels: Ecole des Mines de Douai, Ericsson, INRIA.
	\end{rSubsection}
	\begin{rSubsection}{Projet de fin d'études}{Semestre 4}{ Crédits 5 }
		\item 13 années d'experience dans secteurs très different du écosystème logiciel. 
	\end{rSubsection}
	\begin{rSubsection}{Stage}{Semestre 4}{ Crédits 15 }
		\item 13 années d'experience dans secteurs très different du écosystème logiciel. 
	\end{rSubsection}
\end{rSection}

\end{document}
