

%%%%%%%%%%%%%%%%%%%%%%%%%%%%%%%%%%%%%%%%%
% Medium Length Professional CV
% LaTeX Template
% Version 2.0 (8/5/13)
%
% This template has been downloaded from :
%  http://www.LaTeXTemplates.com
%
% Original author :
% Trey Hunner ( http://www.treyhunner.com/)
%
% Important note :
% This template requires the resume.cls file to be in the same directory as the
% .tex file. The resume.cls file provides the resume style used for structuring the
% document.
%
%%%%%%%%%%%%%%%%%%%%%%%%%%%%%%%%%%%%%%%%%

%----------------------------------------------------------------------------------------
%	PACKAGES AND OTHER DOCUMENT CONFIGURATIONS
%----------------------------------------------------------------------------------------

\documentclass{resume} % Use the custom resume.cls style
\usepackage[utf8]{inputenc}
\usepackage[left=0.75in,top=0.6in,right=0.75in,bottom=0.6in]{geometry} % Document margins
\usepackage{hyperref}
\usepackage{footnote}



\name{Fiches de mission} % Your name

	\address{28 rue Clovis Hugues \\ Lille, 59800 } % Your address
	\address{(+33)~$\cdot$~7~$\cdot$~83~$\cdot$~14~$\cdot$~45~$\cdot$~35 \\ santiagobragagnolo@gmail.com} % Your phone number and
	\address{Santiago Bragagnolo } % Your address
\begin{document}



\begin{rSection}{Donn\'{e}es Biographiques}

\begin{tabular}{ @{} >{\bfseries}l @{\hspace{6ex}} l }
	Pronom et Nom & Santiago Pablo Bragagnolo  \\
	Date de naissance & 16 november 1982  \\
	Nationalit\'{e}s & Argentin, Italien  \\
\end{tabular}

\end{rSection}


\begin{rSection}{Contact}

\begin{tabular}{ @{} >{\bfseries}l @{\hspace{6ex}} l }
	Skype & santiago.bragagnolo  \\
	Linkedin & linkedin.com/in/santiagobragagnolo/  \\
	Portfolio & santiagobragagnolo.wordpress.com  \\
	Blog & knowledgeconvergence.wordpress.com  \\
	Github & github.com/sbragagnolo \\
\end{tabular}

\end{rSection}

%----------------------------------------------------------------------------------------
%	MISSION AUFIERO
%----------------------------------------------------------------------------------------


\section{Fiche de Mission I - Aufiero Informatica }

	Point d'entree. Expliquer très généralement la entreprise .
	
	\subsection{Responsabilités}

Décrivez brièvement a l'aide du tableau ci-dessous les taches que vous assurez au sein de cette mission: indiquez à chaque fois la fréquence dans la colonne 3 et précisez le niveau de responsabilité dans la colonne 4 en entourant le chiffre adéquat à l'aide des critères proposés en note.
\begin{table}[!htbp]
\label{my-label}
\begin{tabular}{|l|l|l|l|l}
\cline{1-4}
   & Descriptif des taches &  \% & Niveau de
Responsabilité  &  \\ \cline{1-4}
 A&  &  &  &  \\ \cline{1-4}
 B&  &  &  &  \\ \cline{1-4}
 C&  &  &  &  \\ \cline{1-4}
 D&  &  &  &  \\ \cline{1-4}
 E&  &  &  &  \\ \cline{1-4}
\end{tabular}
\caption{Responsabilités}
\end{table}

Pouvez-vous classer par ordre d'importance ces différentes tâches (par exemple C-D-A-E-B.... )

Niveau de responsabilité
1- de l'application de consignes ou de procédures
2- de l'amélioration ou de l'optimisation de solutions ou de propositions
3- de la conception de programmes ou de la définition de cahiers des charges 
4- de la définition d'orientations ou de stratégies

	\subsection{Relations humaines}
	
		\subsection{Les relations qu'elle vous permet d'entretenir}
Relations hiérarchiques
* De qui recevez-vous vos objectifs, vos instructions ?
* Sous quelle (s) forme (s) ?
* Qui évalue votre travail ?
* Eventuellement à qui donnez-vous des objectifs. des instructions, des consignes ?
* Sous quelle (s) forme (s) ?
* Comment évaluez-vous l'activité de vos collaborateurs
		\subsubsection{Relations horizontales}
* Avec quel (s) service (s) internes êtes-vous en relation pour l'exécution de cette mission ?
* Sous quelle (s) forme (s) ?
		\subsubsection{Relations extérieures} (clients, fournisseurs,...)
* Avec quel (s) partenaire (s) êtes-vous en relation pour l'exécution de cette mission ?
* Sous quelle (s) forme (s) ?
* Avec quelle fréquence ?

	\subsection{Décrivez les principales qualités que vous avez a mobiliser dans cette mission}
(par exemple créativité, initiative, prise de décision, anticipation, esprit de synthèse, négociation, organisation, qualités de relations,....)

	\subsection{Pouvez-vous presenter une situation-problème que vous avez eu a résoudre dans le cadre de cette mission et la façon de dont vous avez procédé?}
	\subsection {Connaissances mobilisées dans cette mission }
A partir de ce que vous venez de décrire (tâches, responsabilités, relations, qualités,...) identifiez les connaissances que vous mobilisez et éventuellement le type de matériels, de documentation que vous utilisez



%----------------------------------------------------------------------------------------
%	MISSION ECOLE DES MINES
%----------------------------------------------------------------------------------------


\section{Fiche de Mission [] - Ecole }

	Point d'entree. Expliquer très généralement la entreprise .
	
	\subsection{Responsabilités}

Décrivez brièvement a l'aide du tableau ci-dessous les taches que vous assurez au sein de cette mission: indiquez à chaque fois la fréquence dans la colonne 3 et précisez le niveau de responsabilité dans la colonne 4 en entourant le chiffre adéquat à l'aide des critères proposés en note.
\begin{table}[!htbp]
\label{my-label}
\begin{tabular}{|l|l|l|l|l}
\cline{1-4}
   & Descriptif des taches &  \% & Niveau de
Responsabilité  &  \\ \cline{1-4}
 A&  &  &  &  \\ \cline{1-4}
 B&  &  &  &  \\ \cline{1-4}
 C&  &  &  &  \\ \cline{1-4}
 D&  &  &  &  \\ \cline{1-4}
 E&  &  &  &  \\ \cline{1-4}
\end{tabular}
\caption{Responsabilités}
\end{table}

Pouvez-vous classer par ordre d'importance ces différentes tâches (par exemple C-D-A-E-B.... )

Niveau de responsabilité
1- de l'application de consignes ou de procédures
2- de l'amélioration ou de l'optimisation de solutions ou de propositions
3- de la conception de programmes ou de la définition de cahiers des charges 
4- de la définition d'orientations ou de stratégies

	\subsection{Relations humaines}
	
		\subsection{Les relations qu'elle vous permet d'entretenir}
Relations hiérarchiques
* De qui recevez-vous vos objectifs, vos instructions ?
* Sous quelle (s) forme (s) ?
* Qui évalue votre travail ?
* Eventuellement à qui donnez-vous des objectifs. des instructions, des consignes ?
* Sous quelle (s) forme (s) ?
* Comment évaluez-vous l'activité de vos collaborateurs
		\subsubsection{Relations horizontales}
* Avec quel (s) service (s) internes êtes-vous en relation pour l'exécution de cette mission ?
* Sous quelle (s) forme (s) ?
		\subsubsection{Relations extérieures} (clients, fournisseurs,...)
* Avec quel (s) partenaire (s) êtes-vous en relation pour l'exécution de cette mission ?
* Sous quelle (s) forme (s) ?
* Avec quelle fréquence ?

	\subsection{Décrivez les principales qualités que vous avez a mobiliser dans cette mission}
(par exemple créativité, initiative, prise de décision, anticipation, esprit de synthèse, négociation, organisation, qualités de relations,....)

	\subsection{Pouvez-vous presenter une situation-problème que vous avez eu a résoudre dans le cadre de cette mission et la façon de dont vous avez procédé?}
	\subsection {Connaissances mobilisées dans cette mission }
A partir de ce que vous venez de décrire (tâches, responsabilités, relations, qualités,...) identifiez les connaissances que vous mobilisez et éventuellement le type de matériels, de documentation que vous utilisez



%----------------------------------------------------------------------------------------
%	MISSION INRIA
%----------------------------------------------------------------------------------------


\section{Fiche de Mission [] - Enterprise }

	Point d'entree. Expliquer très généralement la entreprise .
	
	\subsection{Responsabilités}

Décrivez brièvement a l'aide du tableau ci-dessous les taches que vous assurez au sein de cette mission: indiquez à chaque fois la fréquence dans la colonne 3 et précisez le niveau de responsabilité dans la colonne 4 en entourant le chiffre adéquat à l'aide des critères proposés en note.
\begin{table}[!htbp]
\label{my-label}
\begin{tabular}{|l|l|l|l|l}
\cline{1-4}
   & Descriptif des taches &  \% & Niveau de
Responsabilité  &  \\ \cline{1-4}
 A&  &  &  &  \\ \cline{1-4}
 B&  &  &  &  \\ \cline{1-4}
 C&  &  &  &  \\ \cline{1-4}
 D&  &  &  &  \\ \cline{1-4}
 E&  &  &  &  \\ \cline{1-4}
\end{tabular}
\caption{Responsabilités}
\end{table}

Pouvez-vous classer par ordre d'importance ces différentes tâches (par exemple C-D-A-E-B.... )

Niveau de responsabilité
1- de l'application de consignes ou de procédures
2- de l'amélioration ou de l'optimisation de solutions ou de propositions
3- de la conception de programmes ou de la définition de cahiers des charges 
4- de la définition d'orientations ou de stratégies

	\subsection{Relations humaines}
	
		\subsection{Les relations qu'elle vous permet d'entretenir}
Relations hiérarchiques
* De qui recevez-vous vos objectifs, vos instructions ?
* Sous quelle (s) forme (s) ?
* Qui évalue votre travail ?
* Eventuellement à qui donnez-vous des objectifs. des instructions, des consignes ?
* Sous quelle (s) forme (s) ?
* Comment évaluez-vous l'activité de vos collaborateurs
		\subsubsection{Relations horizontales}
* Avec quel (s) service (s) internes êtes-vous en relation pour l'exécution de cette mission ?
* Sous quelle (s) forme (s) ?
		\subsubsection{Relations extérieures} (clients, fournisseurs,...)
* Avec quel (s) partenaire (s) êtes-vous en relation pour l'exécution de cette mission ?
* Sous quelle (s) forme (s) ?
* Avec quelle fréquence ?

	\subsection{Décrivez les principales qualités que vous avez a mobiliser dans cette mission}
(par exemple créativité, initiative, prise de décision, anticipation, esprit de synthèse, négociation, organisation, qualités de relations,....)

	\subsection{Pouvez-vous presenter une situation-problème que vous avez eu a résoudre dans le cadre de cette mission et la façon de dont vous avez procédé?}
	\subsection {Connaissances mobilisées dans cette mission }
A partir de ce que vous venez de décrire (tâches, responsabilités, relations, qualités,...) identifiez les connaissances que vous mobilisez et éventuellement le type de matériels, de documentation que vous utilisez




\end{document}
