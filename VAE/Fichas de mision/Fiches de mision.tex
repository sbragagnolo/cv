

%%%%%%%%%%%%%%%%%%%%%%%%%%%%%%%%%%%%%%%%%
% Medium Length Professional CV
% LaTeX Template
% Version 2.0 (8/5/13)
%
% This template has been downloaded from :
%  http://www.LaTeXTemplates.com
%
% Original author :
% Trey Hunner ( http://www.treyhunner.com/)
%
% Important note :
% This template requires the resume.cls file to be in the same directory as the
% .tex file. The resume.cls file provides the resume style used for structuring the
% document.
%
%%%%%%%%%%%%%%%%%%%%%%%%%%%%%%%%%%%%%%%%%

%----------------------------------------------------------------------------------------
%	PACKAGES AND OTHER DOCUMENT CONFIGURATIONS
%----------------------------------------------------------------------------------------

\documentclass{resume} % Use the custom resume.cls style
\usepackage[utf8]{inputenc}
\usepackage[left=0.75in,top=0.6in,right=0.75in,bottom=0.6in]{geometry} % Document margins
\usepackage{hyperref}
\usepackage{footnote}



\name{Fiches de mission} % Your name

	\address{28 rue Clovis Hugues \\ Lille, 59800 } % Your address
	\address{(+33)~$\cdot$~7~$\cdot$~83~$\cdot$~14~$\cdot$~45~$\cdot$~35 \\ santiagobragagnolo@gmail.com} % Your phone number and
	\address{Santiago Bragagnolo } % Your address
\begin{document}



\begin{rSection}{Donn\'{e}es Biographiques}

\begin{tabular}{ @{} >{\bfseries}l @{\hspace{6ex}} l }
	Pronom et Nom & Santiago Pablo Bragagnolo  \\
	Date de naissance & 16 november 1982  \\
	Nationalit\'{e}s & Argentin, Italien  \\
\end{tabular}

\end{rSection}


\begin{rSection}{Contact}

\begin{tabular}{ @{} >{\bfseries}l @{\hspace{6ex}} l }
	Skype & santiago.bragagnolo  \\
	Linkedin & linkedin.com/in/santiagobragagnolo/  \\
	Portfolio & santiagobragagnolo.wordpress.com  \\
	Blog & knowledgeconvergence.wordpress.com  \\
	Github & github.com/sbragagnolo \\
\end{tabular}

\end{rSection}

%----------------------------------------------------------------------------------------
%	MISSION AUFIERO
%----------------------------------------------------------------------------------------


\section{Fiche de Mission I - Aufiero Informatica S.R.L. }

	La premier mission que J'ai choisis c'est la mission industriel plus iconique, dans le intervalle mars 2007 à décembre 2010. Cela est la entreprise que m'a forme le plus sur le développement logiciel industriel. Ici J'ai suis évolué mes aptitudes tant les techniques comme les humaines. 
	Aufiero Informatica est une entreprise moderne de taille petite-moyen (dix a quince personnes), avec un organigramme plutôt plat, sans beaucoup des hiérarchies, mais avec beaucoup des projets. 
	
	\subsection{Responsabilités}
\begin{table}[!htbp]
\label{table-aufiero}
\begin{tabular}{|l|l|l|l|l}
\cline{1-4}
   & Descriptif des taches &  \% & Niveau de Responsabilité \footnotemark &  \\ \cline{1-4}
 A& Interaction avec un client permanent & 20\% & 1 2 3 \textbf{4}  &  \\ \cline{1-4}
 B& Conception / Développement sur systèmes d'un client permanent & 30\%&  1 2 3 \textbf{4} &  \\ \cline{1-4}
 C& Administrateur de base de données  & 10\%  & 1 2 \textbf{3} 4  &  \\ \cline{1-4}
 D& Architect et planning d’application sur des nouveaux projets & 25\% & 1 2 \textbf{3} 4 &  \\ \cline{1-4}
 E& Conception / Développement sur des nouveaux projets & 15\% &1 \textbf{2} 3 4   &  \\ \cline{1-4}
\end{tabular}

\caption{Responsabilités}
\end{table}
Dans la table \ref{table-aufiero}  on peut observer les responsabilités possédées a la fin de mon carrière dans la entreprise.
Dans le cadre d'importance a niveaux stratégique de la boite, l'ordre d'importance de mes taches est: A - D - B - C - E. 

\footnotetext{
Niveau de responsabilité
\begin{enumerate} 
	\item de l'application de consignes ou de procédures
	\item de l'amélioration ou de l'optimisation de solutions ou de propositions
	\item de la conception de programmes ou de la définition de cahiers des charges 
	\item de la définition d'orientations ou de stratégies
\end{enumerate}
}

\subsection{Relations humaines}
	\subsection{Les relations qu'elle vous permet d'entretenir}
	Dans ce boite J'ai eu deux équipes des travaille. Un équipe en charge de soutenir les besoins de notre client plus importante (en plus Oceano). L'autre équipe en charge de recevoir les nouveaux projets arrivent (en plus NewDevs)  . 
	Chaque un de ces équipes m'a rendu different relation humaines et different relations hiérarchiques.  
	
	\subsubsection{Relations hiérarchiques} 
	
		Dans l'équipe Oceano, \textbf{J'ai été chef de projet et organisateur des services}. Notre équipe a été en charge de soutenance des systèmes existantes, réseaux, base de données et infrastructure hardware.
		Dans cet équipe \textbf{nous avons défini les objectifs annual et mensuels de service espère de notre boite} dans des reunions stratégiques avec le manager de logistique et achats et le manager des ventes du cliente, \textbf{en regardent des besoins des travailleurs et budget disponible}. Le résultat de ces reunions a été le plan de travaille de chaque mois. Les personnes en charge de évaluer mon travaille ont été \textbf{le manager de logistique, le manager de ventes et le directeur de Aufiero Informatica}. Comme moyen d'evaluation on a utilise des \textbf{reports hebdomadaire en format de courrier électronique, et des resumes mensuelles en format papier}. \textbf{J'ai rendu des objectifs} généraux aux service de soutenance de réseaux et infrastructure hardware (renovation et reparation des ordinateurs, serveurs , imprimantes, etc), sous le format d'ordres de travaille électroniques. Finalement, pour évaluer l'activité de mes collaborateurs, J'ai utilise principalement des \textbf{réunions informels hebdomadaires, retour des travailleurs et aussi des reports mensuelles en format email}.
		
		Dans l'équipe NewDevs, J'ai été architecte d'application. Mon role principale a été prendre les decisions d'architecture et conception general des solutions informatique, suivre son development et garantir son qualité. 
		Dans cette équipe \textbf{mes objectifs ont été défini en conjoint pour manager de projet et les clients}. Mon travaille a été \textbf{évalué périodiquement dans les reunions de sprint et de fin de projet}. 
		Dans cette équipe J'avais pas des  personne en charges, mais \textbf{J'ai été la reference de développement}.


	\subsubsection{Relations horizontales}
		
		Dans l'équipe Oceano, mon travaille a été articulé  avec  les différentes départements de la entreprise client. J
		'ai travaillé en relation horizontale spécialement avec le manager de logistique et achats , le manager de ventes, et, en mineur proportion, avec le manager de marketing. 
		Le but general de notre collaboration est la definition des objectifs et la mis en place de mis-a-jours et nouvelle fonctionnalités. 
		Les moyens d'organisation ont été principalement les réunions et des échange sous le format de courrier électronique. 
		
	\subsubsection{Relations extérieures} 
		Dans l'equipe Oceano, nous avions des relations avec les fournisseurs de stockage de livres (Les depots de livres on été externalisés). Mon lien avec leur fournisseur de stockage a été arriver a la consolidation de stock chaque fin de année commercial .
		
	\subsection{Décrivez les principales qualités que vous avez a mobiliser dans cette mission}
	
		Les principales qualités mobilisées dans cette mission sont creatives, méthodologiques et organisationnelles. 
		Etre en grand parte responsable de mon propre temps et liste de taches a realiser, a été probablement le challenge plus grande. Développeur des habitudes et adopter des methodologies de travaille pour arriver a bon destin est un travaille quotidian, que demande constance, discipline et pour une personne facile a se disperser comme moi, ca a demande aussi beaucoup de créativité pour me maintenir intéresse. 
		
		
		
		
		
		
		
	\subsection{Pouvez-vous presenter une situation-problème que vous avez eu a résoudre dans le cadre de cette mission et la façon de dont vous avez procédé?}
	\subsection {Connaissances mobilisées dans cette mission }
A partir de ce que vous venez de décrire (tâches, responsabilités, relations, qualités,...) identifiez les connaissances que vous mobilisez et éventuellement le type de matériels, de documentation que vous utilisez



%----------------------------------------------------------------------------------------
%	MISSION ECOLE DES MINES
%----------------------------------------------------------------------------------------


\section{Fiche de Mission [] - Ecole }

	Point d'entree. Expliquer très généralement la entreprise .
	
	\subsection{Responsabilités}

Décrivez brièvement a l'aide du tableau ci-dessous les taches que vous assurez au sein de cette mission: indiquez à chaque fois la fréquence dans la colonne 3 et précisez le niveau de responsabilité dans la colonne 4 en entourant le chiffre adéquat à l'aide des critères proposés en note.
\begin{table}[!htbp]
\label{my-label}
\begin{tabular}{|l|l|l|l|l}
\cline{1-4}
   & Descriptif des taches &  \% & Niveau de
Responsabilité  &  \\ \cline{1-4}
 A&  &  &  &  \\ \cline{1-4}
 B&  &  &  &  \\ \cline{1-4}
 C&  &  &  &  \\ \cline{1-4}
 D&  &  &  &  \\ \cline{1-4}
 E&  &  &  &  \\ \cline{1-4}
\end{tabular}
\caption{Responsabilités}
\end{table}

Pouvez-vous classer par ordre d'importance ces différentes tâches (par exemple C-D-A-E-B.... )

Niveau de responsabilité
1- de l'application de consignes ou de procédures
2- de l'amélioration ou de l'optimisation de solutions ou de propositions
3- de la conception de programmes ou de la définition de cahiers des charges 
4- de la définition d'orientations ou de stratégies

	\subsection{Relations humaines}
	
		\subsection{Les relations qu'elle vous permet d'entretenir}
Relations hiérarchiques
* De qui recevez-vous vos objectifs, vos instructions ?
* Sous quelle (s) forme (s) ?
* Qui évalue votre travail ?
* Eventuellement à qui donnez-vous des objectifs. des instructions, des consignes ?
* Sous quelle (s) forme (s) ?
* Comment évaluez-vous l'activité de vos collaborateurs
		\subsubsection{Relations horizontales}
* Avec quel (s) service (s) internes êtes-vous en relation pour l'exécution de cette mission ?
* Sous quelle (s) forme (s) ?
		\subsubsection{Relations extérieures} (clients, fournisseurs,...)
* Avec quel (s) partenaire (s) êtes-vous en relation pour l'exécution de cette mission ?
* Sous quelle (s) forme (s) ?
* Avec quelle fréquence ?

	\subsection{Décrivez les principales qualités que vous avez a mobiliser dans cette mission}
(par exemple créativité, initiative, prise de décision, anticipation, esprit de synthèse, négociation, organisation, qualités de relations,....)

	\subsection{Pouvez-vous presenter une situation-problème que vous avez eu a résoudre dans le cadre de cette mission et la façon de dont vous avez procédé?}
	\subsection {Connaissances mobilisées dans cette mission }
A partir de ce que vous venez de décrire (tâches, responsabilités, relations, qualités,...) identifiez les connaissances que vous mobilisez et éventuellement le type de matériels, de documentation que vous utilisez



%----------------------------------------------------------------------------------------
%	MISSION INRIA
%----------------------------------------------------------------------------------------


\section{Fiche de Mission [] - Enterprise }

	Point d'entree. Expliquer très généralement la entreprise .
	
	\subsection{Responsabilités}

Décrivez brièvement a l'aide du tableau ci-dessous les taches que vous assurez au sein de cette mission: indiquez à chaque fois la fréquence dans la colonne 3 et précisez le niveau de responsabilité dans la colonne 4 en entourant le chiffre adéquat à l'aide des critères proposés en note.
\begin{table}[!htbp]
\label{my-label}
\begin{tabular}{|l|l|l|l|l}
\cline{1-4}
   & Descriptif des taches &  \% & Niveau de
Responsabilité  &  \\ \cline{1-4}
 A&  &  &  &  \\ \cline{1-4}
 B&  &  &  &  \\ \cline{1-4}
 C&  &  &  &  \\ \cline{1-4}
 D&  &  &  &  \\ \cline{1-4}
 E&  &  &  &  \\ \cline{1-4}
\end{tabular}
\caption{Responsabilités}
\end{table}

Pouvez-vous classer par ordre d'importance ces différentes tâches (par exemple C-D-A-E-B.... )

Niveau de responsabilité
1- de l'application de consignes ou de procédures
2- de l'amélioration ou de l'optimisation de solutions ou de propositions
3- de la conception de programmes ou de la définition de cahiers des charges 
4- de la définition d'orientations ou de stratégies

	\subsection{Relations humaines}
	
		\subsection{Les relations qu'elle vous permet d'entretenir}
Relations hiérarchiques
* De qui recevez-vous vos objectifs, vos instructions ?
* Sous quelle (s) forme (s) ?
* Qui évalue votre travail ?
* Eventuellement à qui donnez-vous des objectifs. des instructions, des consignes ?
* Sous quelle (s) forme (s) ?
* Comment évaluez-vous l'activité de vos collaborateurs
		\subsubsection{Relations horizontales}
* Avec quel (s) service (s) internes êtes-vous en relation pour l'exécution de cette mission ?
* Sous quelle (s) forme (s) ?
		\subsubsection{Relations extérieures} (clients, fournisseurs,...)
* Avec quel (s) partenaire (s) êtes-vous en relation pour l'exécution de cette mission ?
* Sous quelle (s) forme (s) ?
* Avec quelle fréquence ?

	\subsection{Décrivez les principales qualités que vous avez a mobiliser dans cette mission}
(par exemple créativité, initiative, prise de décision, anticipation, esprit de synthèse, négociation, organisation, qualités de relations,....)

	\subsection{Pouvez-vous presenter une situation-problème que vous avez eu a résoudre dans le cadre de cette mission et la façon de dont vous avez procédé?}
	\subsection {Connaissances mobilisées dans cette mission }
A partir de ce que vous venez de décrire (tâches, responsabilités, relations, qualités,...) identifiez les connaissances que vous mobilisez et éventuellement le type de matériels, de documentation que vous utilisez




\end{document}
