%%%%%%%%%%%%%%%%%%%%%%%%%%%%%%%%%%%%%%%%%
% Medium Length Professional CV
% LaTeX Template
% Version 2.0 (8/5/13)
%
% This template has been downloaded from:
% http://www.LaTeXTemplates.com
%
% Original author:
% Trey Hunner (http://www.treyhunner.com/)
%
% Important note:
% This template requires the resume.cls file to be in the same directory as the
% .tex file. The resume.cls file provides the resume style used for structuring the
% document.
%
%%%%%%%%%%%%%%%%%%%%%%%%%%%%%%%%%%%%%%%%%

%----------------------------------------------------------------------------------------
%	PACKAGES AND OTHER DOCUMENT CONFIGURATIONS
%----------------------------------------------------------------------------------------

\documentclass{resume} % Use the custom resume.cls style

\usepackage[left=0.75in,top=0.6in,right=0.75in,bottom=0.6in]{geometry} % Document margins
\usepackage{hyperref}



\name{Santiago Bragagnolo} % Your name
	\address{28 Rue Clovis Hugues \\ Lille, 59800 } % Your address
	\address{(+33)~$\cdot$~7~$\cdot$~83~$\cdot$~14~$\cdot$~45~$\cdot$~35 \\ santiagobragagnolo@gmail.com} % Your phone number and

\begin{document}



\begin{rSection}{Biographic data}

\begin{tabular}{ @{} >{\bfseries}l @{\hspace{6ex}} l }
	Complete name & Santiago Pablo Bragagnolo  \\
	Birth date & 16 november 1982  \\
	Nationalities & Argentinian, Italian  \\
	EU eligibility & YES  \\
\end{tabular}

\end{rSection}


\begin{rSection}{Contact}

\begin{tabular}{ @{} >{\bfseries}l @{\hspace{6ex}} l }
	Skype & santiago.bragagnolo  \\
	Linkedin & linkedin.com/in/santiagobragagnolo/  \\
	%Portfolio & santiagobragagnolo.info  \\
	Blog & knowledgeconvergence.wordpress.com  \\
	Github & github.com/sbragagnolo \\
	SmalltalkHub & smalltalkhub.com/\#!/~sbragagnolo \\
\end{tabular}

\end{rSection}


%----------------------------------------------------------------------------------------
%	TECHNICAL STRENGTHS SECTION
%----------------------------------------------------------------------------------------

\begin{rSection}{INRIA Position related: Brief of Domains}
	\begin{tabular}{ @{} >{\bfseries}l @{\hspace{6ex}} l }
		Robotics & 4 years \\
		Internet of things & 2 years \\
		User experience & 11 years \\
		Gamification (Specific way of UX) & 1 year \\
		Software design, quality and maintenance & 15 years  \\
	\end{tabular}	
\end{rSection}

\begin{rSection}{INRIA Position related: Brief of Competences}
	\begin{tabular}{ @{} >{\bfseries}l @{\hspace{6ex}} l }
		Languages & Pharo Smalltalk, Java, Javascript, Scala, C++ \\
		Technologies & Unix, ROS, SQL, NoSQL, REST, JSON, XML, Multiple-cores  \\
		Microcontroller programming  & Arduino, RaspberryPi, ARM Cortex M3 \\ 
		Versioning tools & SVN, GIT, Mercurial \\
		Continuous integration & Jenkins, Travis \\
		Testing & TDD 10 years experience \\
		Testing Tecnologies & SUnit, JUnit, Selenium, ScalaTest \\
		Design Patterns & 9 years experience \\
		Working in multidisciplinary team & 4 years \\
	\end{tabular}
\end{rSection}


\begin{rSection}{Experience Brief}
	\begin{tabular}{ @{} >{\bfseries}l @{\hspace{6ex}} l }
		Software Industry experience & 11 years \\
		Research transference experience  & 4 years \\
		Effective engineering study & 3.5 years \\
		Teaching programming on university & 5 years \\
	\end{tabular}
\end{rSection}


\begin{rSection}{Technical Strengths}
	\begin{tabular}{ @{} >{\bfseries}l @{\hspace{6ex}} l }
		Paradigms & Object Oriented, Functional \\
		Languages & Pharo, Scala, Java, C/C++, Javascript  \\
		Application Middleware & Tomcat, JBoss, Google App Engine \\
		Message Oriented Middleware & DDS, ROS \\
		MVC Frameworks &  Seaside, SpringMVC, Grails, Play2 \\
		Web Client Frameworks & Backbone.js, Underscore.js, JQuery \\
		SQL Databases & MySQL, PostgreSQL, Oracle, SQLServer  \\
		NO-SQL Databases & MongoDB, BigTable, Db4o  \\
		Versioning & Metacello, Monticello, SVN, GIT \\
		Building & Make, Maven, NDK \\
		Processing Architectures & Multithreading, Multiprocesses, Real time \\
		Software Architectures & REST, Service,  Client \& Server, Data Processing (ETL), Actors\\
		Quality/Testing & SUnit, JUnit, Jenkins \\
		IDEs & Pharo, Squeak, Dolphin, Eclipse, Netbeans \\
		Operative system & Ubuntu/Debian Linux, CentOS Linux / Redhat enterprise \\
	\end{tabular}
\end{rSection}

\begin{rSection}{Methodology Strengths}

\begin{tabular}{ @{} >{\bfseries}l @{\hspace{6ex}} l }
	Working methodologies & Agile, KanBan, XP, Pair programming  \\
	Software design & Domain Driven Design  \\
	Quality assurance & Continuous Integration, Continuous Delivery \\
	Development & Test driven development, Behaviour driven development, Lean development \\
\end{tabular}

\end{rSection}

\begin{rSection}{Languages}

\begin{tabular}{ @{} >{\bfseries}l @{\hspace{6ex}} l }
	Spanish & Native \\
	English & Written and spoken fluent competence \\
	French & Written and spoken medium competence \\
\end{tabular}

\end{rSection}



%----------------------------------------------------------------------------------------
%	WORK EXPERIENCE SECTION
%----------------------------------------------------------------------------------------


\begin{rSection}{Currently working on}


\begin{rSubsection}{INRIA}{April 2015 - Present }{ Research Transference Engineer }{Lille, France}
	\item Transference engineering is about applying research from different research teams into industrial domains. 
	\item Actually I'am working with three teams
	\item FUN is a IOT Team.   \url{https://www.inria.fr/en/teams/fun}
		\begin{itemize}
			\item  Self-Organised Multi hop networks over a net of heterogeneous IOT nodes over IOT-lab \url{https://www.iot-lab.info}
			\item Working with RIOT.
			\item Development of the Demo "Firemen". A network of IOT sensing nodes recognise a fire case and sends a robot to solve the problem.
			\item Team contact Julien Vandaele - julien.vandaele@inria.fr
		\end{itemize}
	\item NON-A is an adaptative control team.   \url{https://www.inria.fr/en/teams/non-a}
			\begin{itemize}
				\item Testing and implementation of potential field path planning
				\item Testing and implementation of tangential path planning
				\item Testing and implementation of optimal minimal points path planning		
				\item Zephyr. The intelligent table. A bed table that responds to voice commands.
				\item Robot-cop. A wanderer robot that sends alarms based on the recognition of potential intruders . 	
				\item Team contact Zheng Gang - gang.zheng@inria.fr
			\end{itemize}
		
	\item RMOD is a team that researches about software evolution and maintenance.  \url{https://www.inria.fr/en/teams/rmod}
			\begin{itemize}
				\item Compiling Pharo smalltalk Virtual machine for android platform \url{https://github.com/pharo-project/pharo-vm}
				\item TaskIT. Framework for dealing with concurrency and parallelism. 
				\item Scale. Framework for scripting with pharo. 
				\item Blockchain based application analysis. 
				\item Team contact Stephane Ducasse - stephane.ducasse@inria.fr
			\end{itemize}
	\item Contact reference: Sylvain Karpf; sylvain.karpf@inria.fr

\end{rSubsection}



	\begin{rSubsection}{Spare time}{October 2014 - Currently }{Mako/Makros} {Lille, France}
		\item Component oriented framework for general programming and for  quick prototyping and reproducibility of robotics behaviour
		\item Support for Pharo 3.0, 4.0 and 5.0.
		\item https://github.com/sbragagnolo/Mako
		\item https://github.com/sbragagnolo/Makros
	\end{rSubsection}

	\begin{rSubsection}{Spare time}{October 2014 - Currently }{MetaDDS/SimpleDDS/ROSDDS} {Lille, France}
		\item Skeleton and canon implementation of a Data Delivery Service (publisher/subscriber) framework and it full ROS and Pharo smalltalk specific implementation. Based on ROS standard and OMG-DDS standard.
		\item Support for Pharo 3.0, 4.0 and 5.0.
		\item https://github.com/sbragagnolo/MetaDDS
		\item https://github.com/sbragagnolo/SimpleDDS
		\item https://github.com/sbragagnolo/ROSDDS
	\end{rSubsection}

	\begin{rSubsection}{Spare time}{Feburary 2013 - Currently }{TaskIT}{Lille, France}
		\item Support for Pharo 3.0, 4.0 and 5.0.
		\item Framework for managing concurrency and parallel computations.
		\item Based in future/promises and it composition for synchronisation of processing. 
		\item https://github.com/sbragagnolo/taskit
	\end{rSubsection}
	
\end{rSection}


\begin{rSection}{Industry and Research Experience}
		
\begin{rSubsection}{Ericsson}{May 2014 - October 2014 }{ Senior Java and C++ developer }{Malaga, Spain}
	\item Design and development of a troubleshooting solution for mobile telephone networks
	\item ERA (Ericsson ran analyzer) Project. C++ and Java based high-performance client, based on data analysis and complex visualizations over a GIS Map.
	\item TPS (Trace processor server) Project. Java high-performance server for trace processing (system that faces problems of teras of data) with some IronPython for processing text files, with part of processing over hadoop and hive.
	\item ODG (OSS Data gateway) Project. Project that analyze data from OSS servers. Also Java high-performance server. 
	\item All of these project are highly Multithreaded and based on distributed processing.  
	\item Most of the data processed is reduced and stored into the client mapping data from several servers with workflow processing. (Based in ETL pattern)
	\item TDD and Continuous delivery evangalizer.
	\item Working under Agile with Scrum methodology
	\item Contact reference: Jose Antonio Hurtado +34 670 42 66 01; joseantonio.hurtado@gmail.com
	
\end{rSubsection}


\begin{rSubsection}{Ecole des mines de Douai}{September 2012 - February 2014}{Software Research Engineer, Robotics}{Douai, Nord-Pas-de-Calais}
	\item Design and development of the project RoboShop (http://car.mines-douai.fr/RoboShop), a robotic system for aiding persons to navigate into unknown spaces.
	\item My responsibilities are planning, engineering, development and technologic report writing for the projects RoboShop and PhaROS (http://car.mines-douai.fr/PhaROS). 
	\item PhaROS is a framework for developing robot solutions for Pharo Smalltalk on the robotic middleware ROS. It implements a real time distributed architecture.
	\item The project RoboShop was presented in Picom and Vad Conext 2013 (Some content about http://car.mines-douai.fr/2013/11/roboshop-demo-16oct2013/) 
	\item The project PhaROS is going to be presented in FOSDEM 2014, (https://fosdem.org/2014/) as PhaROS Towards Live Environments in Robotics in the Smalltalk devroom.
	\item From this position i also contribute with the project TaskIT (http://smalltalkhub.com/\#!/~sbragagnolo/TaskIT), which is a project for managing parallelism and concurrence in a Pharo smalltalk environment.
	\item The used technologies for the robot side are: Pharo, ROS, Python and C++. For the graphical interface: Pharo, Seaside, Bootstrap, Javascript, html+css3, for iPad
		\item Contact reference: Noury Bouraqadi +33 6 27 07 48 42; noury.bouraqadi@mines-douai.fr
\end{rSubsection}


%------------------------------------------------

\begin{rSubsection}{Fanwards (http://www.fanwards.com/)}{November 2011 - August 2012}{Software engineer, gamification}{Ciudad de Buenos Aires, Argentine}
	\item Design and implementation of the web application Fanwards, front and backend. Working with the CTO.
	\item Using for front end heavy weight clients based on javascript technologies such as Backbone.js, jquery, underscore.js , mustache and for the view HTML5 and CSS3.
	\item Using for the backend a Google App Engine (GAE) Server with Scala \& Java, Objectify for the mapping between objects and google's BigTable, Spring MVC and RestFul frameworks for routing and dispatching of exposed and scheduled behaviours.Finally twitter4j and facebookRest for interacting with social networks.  
	\item From this position i have developed: a full intelligent single-page client based on javascript,  a small functional library for javascript implementing some of the common haskell features (partial application, curryfication and function compositions), really useful abstractions for AJAX request processing, also developed a social network crawler for analysing users comments for branding (with heuristics to analyse the meaning of each comment). 
	\item From this position i also participate not just in software design, planning and architecture but also in the gamification process of the application.
	\item During all the development of the application we used Scala BDD and TDD techniques with great success.
	\item Crossing data for brand reports with Map reduce. 
	\item You can watch the most important part of the implementation in the following example 		
	\url{https://www.facebook.com/pages/Melee-Island-Inc/252598398140724?id=252598398140724&sk=app_232320516837452}
	
	\item Contact reference: Claudio Fernandez claudiof@gmail.com
\end{rSubsection}



\begin{rSubsection}{Aufiero Informatica}{March 2011 - November 2011}{Software architect, designer \& developer}{Ciudad de Buenos Aires, Argentine}
	\item Project manager in charge of a 4 persons team.
	\item Reseller/Partner management system based on Groovy on Grails technology and communicating to software legacy done in PHP.
	\item Mail campaign web system for internal usage (For AVG campaigns). Done in Groovy and Grails / Jasper reports
	\item Email send system multi-engine, auto-deployable with load balance and mail tracking. Done in Groovy and Java, using Apache Email. 
	\item Single sign on system for our different platforms - Done in Groove and grails.
	\item Contact Reference: Osvaldo Aufiero osvaldo@aufiero.com.ar
\end{rSubsection}


\begin{rSubsection}{Buscouniversidad .com}{January 2011 -  March - 2011}{Developer \& system designer}{Ciudad de Buenos Aires, Argentine}
	\item Designing of Data base structure, SQL queries. 
	\item Design and development of the directory system (such as OLX, Craiglists, etc, but specific for universities)
	\item Based on PHP with Zend framework, sphinx and Javascript with JQuery. And in the development of processing tools
	\item Design and development of a processing based on patterns for recognising rejects and angry people (python);
	\item Design and development of an easy code generator for Zend framework
	\item Contact Reference: Pablo Morales pablofmorales@gmail.com
\end{rSubsection}

\begin{rSubsection}{Aufiero Informatica}{March 2007 - December 2010}{Software designer \& developer - Project manager }{Ciudad de Buenos Aires, Argentine}
	\item Designer and developer of an Accountant management system - Java, Flex 3.1 JBoss 
	\item Maintainer, developer and DBA of an Editorial integral management system (stock, accounting, finances, sells, shopping, etc) - visual basic 6.0 / sql server / Crystal reports.
	\item Designer and developer of ETL pattern for making up a small datawarehouse for sales analysis.
	\item Designer and developer - Classified ads management system - Net Framework 3.0 C\# Nhibernate Windows form
	\item Software designer, developer and functional analysis - Billing online system - PHP, CakePHP, Javascript jquery. 
	\item Software designer, developer - Electronic Invoice system (Based on the local taxes system) - Java, Groovy and Grails / jasper reports.
	\item Contact Reference: Osvaldo Aufiero osvaldo@aufiero.com.ar
\end{rSubsection}


\begin{rSubsection}{MSA}{2006}{DBA Oracle 9i \& Postgres SQL}{Ciudad de Buenos Aires, Argentine}
	\item DBA of three productive databases related with ticket system (kind of ticketek but with less stress) 
	\item My tasks were Data base monitoring, Backup, SQL Security, Database and query tuning for the productive systems (Oracle 9i) and making up configurations for eventual projects (Usually Postgres SQL)
	\item Parallely i had some small responsibilities in eventual projects 
\end{rSubsection}


\begin{rSubsection}{Research for decision}{2003 - 2005}{Developer}{Ciudad de Buenos Aires, Argentine}
	\item Design and development of polls in eole/saxophone (language and poll system).
	\item Servers, machines and network maintenance.
\end{rSubsection}

\begin{rSubsection}{Freelance}{2002 - 2009}{Developer, Designer, Architect, DBA \& Client management}{Ciudad de Buenos Aires, Argentine}
	\item This are the projects i have developed as independent, several of them still in usage, and I maintained almost them for one to two years. 
	\item  November 2002 to July 2003 - Nutritional preparations management system for a Parenteral Laboratory (UNANUT) -  Visual Basic \& Microsoft Access ( Supported until 2005). Contact Reference: Lorena Fazio faziolorena@yahoo.com.ar
	\item  June 2003 to August 2003 - Stock management system for a box factory  - Visual Basic \& Microsoft Access ( Supported until 2004). Contact Reference: Gerardo Grimaldi grimaldi.gerardo@gmail.com
	\item  July 2009 to March 2010 - Certification and courses management system SFAP -  PHP (http://www.facpce.org.ar/)  (Supported until 2011). Contact Reference: Christian Milokanovich milo76@gmail.com
\end{rSubsection}



\end{rSection}


%----------------------------------------------------------------------------------------
%	OPEN SOURCE SECTION
%----------------------------------------------------------------------------------------
%------------------------------------------------
\begin{rSection}{Experience in open source - community projects}

\begin{rSubsection}{On my own}{Mars - May 2014}{Making work Scala + Play + MongoDB}{Malaga}
	\item Putting together some existant technology for a project on my own
	\item Fully developed with Scala and Play. It has some java dependancies (Jackson for JSon marshalling)
	\item https://github.com/sbragagnolo/mongodb
\end{rSubsection}

\begin{rSubsection}{On my own}{June - August 2014}{Social secure plugin for Play running on top of MongoDB}{Malaga}
	\item Social secure backend implementation for running on mongoDB
	\item Fully developed with Scala and Play. 
	\item https://github.com/sbragagnolo/SocialSecurePlayMongo
\end{rSubsection}

\begin{rSubsection}{On my own}{Mars - August 2014}{Android }{Malaga}
	\item Android application for knowing where to go to play football.
	\item https://github.com/sbragagnolo/Where
\end{rSubsection}


\begin{rSubsection}{Google summer of code}{May 2012 - September 2012}{Type inference on dynamic languages}{Ciudad de Buenos Aires, Argentine}
	\item Design, development, planning and research in the topic of Concrete type inference. 		
	\item Mentored by Nicolas Passerini (npasserini@gmail.com).
	\item The proposal is registered http://gsoc2012.esug.org/projects/type-inference 
	\item From this project i have implemented a concrete type inference system for Pharo Smalltalk 1.4, a graph of methods to be executed as response of the analysis of a given expression,  and an object oriented logger. 
	\item I also took the work of blogging all the work progress and design decisions in a blog. This project was presented in the conference ESUG 2012 - Gent - (http://www.esug.org/wiki/pier/Conferences/2012) 
	\item Site - http://concretetypeinference.blogspot.fr/. 
	\item Concrete type inferencer and the call graph analyser (Kwisatz Haderach) - http://ss3.gemstone.com/ss/ConcreteTypeInference.html
	\item Paule le poulpe; Object oriented Logger - http://smalltalkhub.com/mc/sbragagnolo/PLP/main
\end{rSubsection}

 
\begin{rSubsection}{ESUG financed Project}{ May 2011 - September 2011}{Developer in DBXTalk}{Ciudad de Buenos Aires, Argentine}
	\item Design and implementation of the automatic binding and scaffolding for DBXTalk (ex SqueakDBX) and Glorp.
	\item DBXTalk (http://dbxtalk.smallworks.com.ar/) is a bridge that gives support for the  database systems to mainstream Squeak and Pharo smalltalk. 
	\item This project with the porting of Glorp to Pharo (by Guillermo Polito) were presented in ESUG 2011 - Edimburgh (http://www.esug.org/wiki/pier/Conferences/2011)
	\item Contact Reference: Guillermo Polito guillermopolito@gmail.com
\end{rSubsection}

\begin{rSubsection}{FACPCE financed Project - Used for SFAP }{ 2009 }{Developer in Cornucopia framework}{Ciudad de Buenos Aires, Argentine}	
	\item PHP Cornucopia. Is an already obsolete full stack framework. It provides:
	\begin{itemize}
		\item Collections
		\item Simple ORM configurable by metadata 
		\item Simple dependency injector
		\item HTML reification for composing and HTML generation
		\item Javascript generation by metadata (Requirements, etc)
		\item Request reification as object
		\item Session reification as object
	\end{itemize}
	\item Check it out from: https://github.com/sbragagnolo/cornucopia
	\item Contact Reference: Christian Milokanovich milo76@gmail.com
\end{rSubsection}


\begin{rSubsection}{On my own for an university topic}{ 2007 }{Developer in C Objects framework}{Ciudad de Buenos Aires, Argentine}
	\item C Objects. Is a framework to provide several object oriented programming features to C. It provides:
	\begin{itemize}
		\item Collections (Dictionary, List) 
		\item Strings
		\item Server, with node objects for managing requests
		\item Automatons for simple language interpretation 
		\item Threads reification 
		\item Mutex and Conditional reifications
		\item Error management (managed by signals, giving the change to register error handlers and to raise errors) 
		\item Memory management.
	\end{itemize}
	\item Checkit out from: https://github.com/sbragagnolo/c-objetos
	\item Contact Reference: nicolassouto@gmail.com
\end{rSubsection}


\end{rSection}

%----------------------------------------------------------------------------------------
%	TEACHING SECTION
%----------------------------------------------------------------------------------------

\begin{rSection}{Teaching experience}


\begin{rSubsection}{Universidad Tecnologica Nacional (UTN)}{March 2007 - July 2012}{Adhonorem teaching assistant at Advanced programming techniques}{Ciudad de Buenos Aires, Argentine}
\item Teaching the next concepts, techniques and tools
\begin{itemize}
	\item Object oriented programming
	\begin{itemize}
		\item Patterns
		\item Methodologies TDD/BDD, DDD, Agile
		\item Re-factors
		\item Meta-programming
	\end{itemize}
	\item Basic architectures
	\item Technologies
	\begin{itemize}
		\item Maven 
		\item IDEs - Eclipse, Idea, Netbeans
		\item JUnit, ScalaTest
		\item SVN \& GIT
	\end{itemize}
	\item Dynamic languages 
	\begin{itemize}
		\item  Scala
		\item  Python
		\item  Smalltalk
		\item  Self
	\end{itemize}
	\item Modern applied concepts
	\begin{itemize}
		\item Traits and Mixins
		\item  Lambdas / anonymous functions
	\end{itemize}
\end{itemize}
\item Contact Reference: npasserini@gmail.com
\end{rSubsection}

\begin{rSubsection}{Universidad Tecnologica Nacional (UTN)}{March 2007 - December 2011}{Adhonorem teaching assistant at Programming paradigms}{Ciudad de Buenos Aires, Argentine}
\item Teaching the next concepts, techniques and tools
\begin{itemize}
	\item Object oriented paradigm
	\begin{itemize}
		\item Pharo Smalltalk
	\end{itemize}
	\item Functional oriented paradigm
	\begin{itemize}
		\item GHC
		\item WinHugs
	\end{itemize}
	\item Logical oriented paradigm 
	\begin{itemize}
		\item  Swi prolog
	\end{itemize}
\end{itemize}
\item Contact Reference: carlombardi@gmail.com
\end{rSubsection}


\begin{rSubsection}{Fundacion Proydesa}{March 2006 - November 2007}{Oracle DBA Instructor}{Ciudad de Buenos Aires, Argentine}
	\item In this foundation i worked as instructor of three of the four basic modules of Oracle for Database administration
	\item SQL (Module 1)
	\item Engine Architecture (Module 2)
	\item Tuning (Module 4)
	\item I also participate as instructor in the instruction of new instructors for modules 1 and 2.
\end{rSubsection}


\end{rSection}



%----------------------------------------------------------------------------------------
%	EDUCATION SECTION
%----------------------------------------------------------------------------------------

\begin{rSection}{Education}


{\bf Universidad Tecnologica Nacional (UTN) - Ciudad de Buenos Aires - Argentine} \hfill {\em January 2004 - Aborted on 2011 because of personal problems, having papers to proof 7 semesters of career of 10 } \\ 
Software Engineer  \\

{\bf Ing. Otto Krause - Ciudad de Buenos Aires - Argentina} \hfill {\em December 2001} \\ 
Technic in computation \\
\end{rSection}

%----------------------------------------------------------------------------------------
%	EXAMPLE SECTION
%----------------------------------------------------------------------------------------

%\begin{rSection}{Section Name}

%Section content\ldots

%\end{rSection}

%----------------------------------------------------------------------------------------

\end{document}
