%%%%%%%%%%%%%%%%%%%%%%%%%%%%%%%%%%%%%%%%%
% Medium Length Professional CV
% LaTeX Template
% Version 2.0 (8/5/13)
%
% This template has been downloaded from:
% http://www.LaTeXTemplates.com
%
% Original author:
% Trey Hunner (http://www.treyhunner.com/)
%
% Important note:
% This template requires the resume.cls file to be in the same directory as the
% .tex file. The resume.cls file provides the resume style used for structuring the
% document.
%
%%%%%%%%%%%%%%%%%%%%%%%%%%%%%%%%%%%%%%%%%

%----------------------------------------------------------------------------------------
%	PACKAGES AND OTHER DOCUMENT CONFIGURATIONS
%----------------------------------------------------------------------------------------

\documentclass{resume} % Use the custom resume.cls style

\usepackage[left=0.75in,top=0.6in,right=0.75in,bottom=0.6in]{geometry} % Document margins
\usepackage{hyperref}



\name{Santiago Bragagnolo} % Your name
	\address{14 Dennis du Peage \\ Lille, 59800 } % Your address
	\address{(+33)~$\cdot$~7~$\cdot$~83~$\cdot$~14~$\cdot$~45~$\cdot$~35 \\ santiagobragagnolo@gmail.com} % Your phone number and

\begin{document}



\begin{rSection}{Donn\'{e}es Biographiques}

\begin{tabular}{ @{} >{\bfseries}l @{\hspace{6ex}} l }
	Pronom et Nom & Santiago Pablo Bragagnolo  \\
	Date de naissance & 16 november 1982  \\
	Nationalit\'{e}s & Argentin, Italien  \\
	Admissible dans EU & OUI  \\
\end{tabular}

\end{rSection}


\begin{rSection}{Contact}

\begin{tabular}{ @{} >{\bfseries}l @{\hspace{6ex}} l }
	Skype & santiago.bragagnolo  \\
	Linkedin & linkedin.com/in/santiagobragagnolo/  \\
	Portfolio & santiagobragagnolo.info  \\
	Blog & knowledgeconvergence.wordpress.com  \\
	Github & github.com/sbragagnolo \\
	SmalltalkHub & smalltalkhub.com/\#!/~sbragagnolo \\
\end{tabular}

\end{rSection}


%----------------------------------------------------------------------------------------
%	TECHNICAL STRENGTHS SECTION
%----------------------------------------------------------------------------------------

\begin{rSection}{INRIA li\'{e}e \`{a} la position: Br\`{e}ve des Domaines}
	\begin{tabular}{ @{} >{\bfseries}l @{\hspace{6ex}} l }
		Robotique & 2 ann\'{e}es \\
		Internet of things & 2 ann\'{e}es \\
		User experience (Interactions Homme Machine) & 12 ann\'{e}es \\
		Gamification (Interactions Homme Machine particulier) & 1 ann\'{e}e \\
		G\'{e}nie logiciel, qualit\'{e} et maintenance logicielle & 12 ann\'{e}es  \\
	\end{tabular}	
\end{rSection}

\begin{rSection}{INRIA li\'{e}e \`{a} la position: Br\`{e}ve des Comp\'{e}tences}
	\begin{tabular}{ @{} >{\bfseries}l @{\hspace{6ex}} l }
		Langages & Pharo Smalltalk, Java, Javascript, Scala, C++ \\
		Technologies & Unix, ROS, SQL, NoSQL, REST, JSON, XML, Multiple-cores \\
		D\'{e}veloppement sur Microcontroleur  & Arduino \\ 
		Gestion de version & SVN, GIT, Mercurial \\
		L‘int\'{e}gration continue & Jenkins \\
		Tests & TDD 8 years experience \\
		Tecnologies du Test & SUnit, JUnit, Selenium, ScalaTest \\
		Design Patterns & 7 years experience \\
		Exp\'{e}rience en \'{e}quipe pluridisciplinaire & 2 years \\
	\end{tabular}
\end{rSection}


\begin{rSection}{Experience Brief}
	\begin{tabular}{ @{} >{\bfseries}l @{\hspace{6ex}} l }
		Exp\'{e}rience dans l'industrie du logiciel & 11 years \\
		Exp\'{e}rience dans la recherche (Robotique) & 2 years \\
		Temps effectif de l'\'{e}tude de l'ing\'{e}nierie & 3.5 years \\
		Enseigner la programmation sur l'universit\'{e} & 5 years \\
	\end{tabular}
\end{rSection}


\begin{rSection}{Technical Strengths}
	\begin{tabular}{ @{} >{\bfseries}l @{\hspace{6ex}} l }
		Paradigmes & Object Oriented, Functional \\
		Langages & Pharo, Scala, Java, C/C++, Javascript  \\
		Application Middleware & Tomcat, JBoss, Google App Engine \\
		Message Oriented Middleware & DDS, ROS \\
		MVC Frameworks &  Seaside, SpringMVC, Grails, Play2 \\
		Frameworks du Client Web & Backbone.js, Underscore.js, JQuery \\
		SQL Bases de donn\'{e}es & MySQL, PostgreSQL, Oracle, SQLServer  \\
		NO-SQL Bases de donn\'{e}es  & MongoDB, BigTable, Db4o  \\
		Gestion de versions & Metacello, Monticello, SVN, GIT \\
		Building & Make, Maven \\
		Architectures de traitement & Multithreading, Multiprocesses, Real time \\
		Architecture Logicielle & REST, Service,  Client \& Server, Data Processing (ETL), Actors\\
		Qualit\'{e}/Test & SUnit, JUnit, Jenkins \\
		IDEs & Pharo, Squeak, Dolphin, Eclipse, Netbeans \\
		Syst\`{e}me d'Exploitation & Ubuntu/Debian Linux, CentOS Linux / Redhat enterprise \\
	\end{tabular}
\end{rSection}

\begin{rSection}{Methodology Strengths}

\begin{tabular}{ @{} >{\bfseries}l @{\hspace{6ex}} l }
	
	M\'{e}thodes de travail & Agile, KanBan, XP, Pair programming  \\
	Conception de logiciels & Domain Driven Design  \\
	Assurance de la qualit\'{e} & Continuous Integration, Continuous Delivery \\
	D\'{e}veloppement & Test driven development, Behaviour driven development, Lean development \\
\end{tabular}

\end{rSection}

\begin{rSection}{Langages}

\begin{tabular}{ @{} >{\bfseries}l @{\hspace{6ex}} l }
	Espagnol & natif \\
	Anglais & \'{E}crit et parl\'{e} couramment comp\'{e}tence \\
	French & \'{E}crit et parl\'{e} comp\'{e}tence moyenne \\
\end{tabular}

\end{rSection}



%----------------------------------------------------------------------------------------
%	WORK EXPERIENCE SECTION
%----------------------------------------------------------------------------------------


\begin{rSection}{Travail en cours}

	\begin{rSubsection}{On my own}{D\'{e}cembre 2014 - Actuellement }{Robots-du-Nord project}
		\item Robots du Nord est un projet sur le d\'{e}veloppement des robots. La premi\`{e}re \'{e}tape de ce projet est de g\'{e}n\'{e}rer du contenu sur la fa\c{c}on de concevoir et d\'{e}velopper des mat\'{e}riels et logiciels pour la robotique et la domotique.
		\item Pre startup prototypage
		\item Business model
		\item R\'{e}daction de contenu
	\end{rSubsection}
	
	\begin{rSubsection}{On my own}{Octobre 2014 - Actuellement }{MetaDDS/SimpleDDS}
		\item Squelette et impl\'{e}mentation par d\'{e}faut d'un Framework pour Data Delivery Service.  Bas\'{e} sur la norme ROS et la norme OMG-DDS.		
		\item d\'{e}velopp\'{e} dans Pharo 3.0 
		\item http://smalltalkhub.com/\#!/~sbragagnolo/MetaDDS
		\item http://smalltalkhub.com/\#!/~sbragagnolo/SimpleDDS
%	\end{itemize}
	\end{rSubsection}

	\begin{rSubsection}{On my own}{F\'{e}vrier 2014 - Actuellement }{TaskIT}
		\item Maintien et d\'{e}veloppement de TaskIT avec Guillermo Polito
		\item TaskIT est un Framework de traitement, orient\'{e} objet, facile \`{a} utiliser, puissant, bien test\'{e}e, simple et fiable. D\'{e}velopp\'{e} pour r\'{e}pondre aux exigences d'un environnement de traitement en temps r\'{e}el.
		\item http://smalltalkhub.com/mc/sbragagnolo/TaskIT/main
	\end{rSubsection}

\end{rSection}


\begin{rSection}{Exp\'{e}rience dans l'industrie et dans la recherche}

\begin{rSubsection}{Ericsson}{Mai 2014 - Septembre 2014 }{ D\'{e}veloppeur confirme Java et C++  }{Malaga, Espagne}
	\item Conception et d\'{e}veloppement d'une solution de d\'{e}pannage pour les r\'{e}seaux de t\'{e}l\'{e}phonie mobile
	\item ERA (Ericsson ran analyzer) Projet. Client haute performance fait en C ++ et Java. Analyse des donn'{e}es et visualisations complexes sur une carte SIG.
	\item TPS (Trace processor server) Projet. Java serveur de haute performance pour le traitement de trace (syst\`{e}me qui fait face \`{a} des probl\`{e}mes de teras de donn\'{e}es). Hadoop et Hive.
	\item ODG (OSS Data gateway) Projet. Java serveur de haute performance pour le traitement des archives de configuration de serveurs OSS.
	\item \'{E}vang\'{e}lisateur de TDD et la livraison continue.
	\item Agile, scrum
	\item R\'{e}f\'{e}rence de contact: Jose Antonio Hurtado +34 670 42 66 01; joseantonio.hurtado@gmail.com
	
\end{rSubsection}


\begin{rSubsection}{Ecole des mines de Douai}{Septembre 2012 - F\'{e}vrier 2014}{Ing\'{e}nieur de Recherche de logiciels, robotique}{Douai, Nord-Pas-de-Calais}
	\item Conception et d\'{e}veloppement de l'RoboShop de projet (http://car.mines-douai.fr/RoboShop), un syst\`{e}me robotique pour aider les personnes \`{a} naviguer dans des espaces inconnus.
	\item Mes responsabilit\'{e}s sont la planification, l'ing\'{e}nierie, le d\'{e}veloppement et le rapport \'{e}crit technologique pour les projets RoboShop et Pharos (http://car.mines-douai.fr/PhaROS).
	\item PhaROS est un Framework pour le d\'{e}veloppement de solutions robotiques pour Pharo Smalltalk sur le middleware robotique ROS. Il met en oeuvre une architecture distribu\'{e}e en temps r\'{e}el.
	\item Le projet a \'{e}t\'{e} pr\'{e}sent\'{e} en RoboShop Picom et Vad Conext 2013 (Certains contenus propos http://car.mines-douai.fr/2013/11/roboshop-demo-16oct2013/)
	\item Le projet Pharos va etre pr\'{e}sent\'{e} dans FOSDEM 2014, (https://fosdem.org/2014/) Pharos Vers environnements vivants en Robotique dans la salle d\'{e}di\'{e}e \`{a} Smalltalk.
	\item De cette position, je contribue aussi avec le TaskIT de projet (http://smalltalkhub.com/\#!/~sbragagnolo/TaskIT), qui est un projet pour g\'{e}rer le parall\'{e}lisme et la concurrence dans un environnement Smalltalk Pharo.
	\item Les technologies utilis\'{e}es pour le côt\'{e} du robot sont: Pharo, ROS, Python et C ++. Pour l'interface graphique: Pharo, Seaside, Bootstrap, Javascript, HTML + CSS3, pour iPad
		\item R\'{e}f\'{e}rence de contact: Noury Bouraqadi +33 6 27 07 48 42; noury.bouraqadi@mines-douai.fr
\end{rSubsection}


%------------------------------------------------

\begin{rSubsection}{Fanwards (http://www.fanwards.com/)}{Novembre 2011 - Aout 2012}{Ing\'{e}nieur logiciel confirme, gamification}{Ciudad de Buenos Aires, Argentine}
	\item Conception et mise en œuvre de la Fanwards d'applications Web, avant et backend. Travailler avec le CTO.
	\item Utilisation pour frontaux clients de poids lourds bas\'{e}s sur des technologies telles que Backbone.js javascript, jquery, underscore.js, moustache et pour le vue HTML5 et CSS3.
	\item Nous avons utilis\'{e} pour le backend de Google App Engine (GAE) Server avec Scala \& Java, Objectify pour le mappage entre les objets et de Google BigTable, Spring MVC et cadres reposante pour le routage et l'exp\'{e}dition des expos\'{e}s et pr\'{e}vue twitter4j behaviours.Finally et facebookRest pour interagir avec les r\'{e}seaux sociaux.
	\item De cette position, je ai d\'{e}velopp\'{e}: un client single-page bas\'{e} sur JavaScript, une petite biblioth\`{e}que fonctionnelle pour javascript avec certaines caract\'{e}ristiques de Haskell (compositions application partielle, curryfication et de fonction), abstractions vraiment utiles pour AJAX traitement de la demande, aussi d\'{e}velopp\'{e} un robot de r\'{e}seau social pour analyser les commentaires des utilisateurs pour la marque (avec heuristiques pour analyser la signification de chaque commentaire).
	
	\item Pendant tout le d\'{e}veloppement de l'application nous avons eu recours \`{a} des techniques Scala BDD et TDD avec beaucoup de succ\`{e}s.
	
	\item R\'{e}f\'{e}rence de contact: Claudio Fernandez Claudio.Fernandez@Point72.com
\end{rSubsection}



\begin{rSubsection}{Aufiero Informatica}{Mars 2011 - Novembre 2011}{Architecte logiciel, concepteur et d\'{e}veloppeur}{Ciudad de Buenos Aires, Argentine}
	\item Chef de projet en charge d'une \'{e}quipe de 4 personnes.
	\item Syst\`{e}me de gestion Revendeur / Partenaire bas\'{e}e sur Groovy Grails sur la technologie et la communication \`{a} l'h\'{e}ritage du logiciel fait en PHP.
	\item Campagne de publipostage du syst\`{e}me Web \`{a} usage interne (Pour les campagnes AVG). Fait en Groovy et Grails et Jasper report
	\item Syst\`{e}me \'{e}metteur-mail multi-moteur, auto-d\'{e}ployable avec \'{e}quilibrage de charge et de suivi du courrier. Fait en Groovy et Java, en utilisant Apache Email.
	\item Syst\`{e}me Single sign on -  Fait en Groovy et Grails.
	\item R\'{e}f\'{e}rence de contact: Osvaldo Aufiero osvaldo@aufiero.com.ar
\end{rSubsection}


\begin{rSubsection}{Buscouniversidad .com}{January 2011 -  March - 2011}{D\'{e}veloppeur et concepteur du syst\`{e}me}{Ciudad de Buenos Aires, Argentine}
	\item Concevoir la structure de base de donn\'{e}es. Requetes SQL.
	\item Conception et d\'{e}veloppement du syst\`{e}me de r\'{e}pertoire (comme OLX, Craigslists, etc, mais qui sont sp\'{e}cifiques pour les universit\'{e}s)
	\item Fait en PHP avec Zend Framework, sphinx et Javascript avec JQuery. Et dans le d\'{e}veloppement d'outils de traitement
	\item Conception et d\'{e}veloppement d'un traitement bas\'{e} sur des mod\`{e}les de reconnaissance rejets et des gens en col\`{e}re (python)
	\item Conception et d\'{e}veloppement d'un g\'{e}n\'{e}rateur de code facile pour Zend Framework
	\item R\'{e}f\'{e}rence de contact: Pablo Morales pablofmorales@gmail.com
\end{rSubsection}

\begin{rSubsection}{Aufiero Informatica}{March 2007 - December 2010}{Concepteur et d\'{e}veloppeur de logiciels - Chef de projet}{Ciudad de Buenos Aires, Argentine}
	\item Concepteur et d\'{e}veloppeur d'un syst\`{e}me de gestion de comptable - Java, Flex 3.1 JBoss
	\item Mainteneur, d\'{e}veloppeur et DBA d'un syst\`{e}me \'{e}ditorial int\'{e}gr\'{e} de gestion (stocks, la comptabilit\'{e}, les finances, vend, shopping, etc.) - de base serveur visuelle 6.0 / sql / Crystal Reports.
	\item Concepteur et d\'{e}veloppeur d'un ETL pattern pour faire un petit entrepôt de donn\'{e}es pour l'analyse des ventes.
	\item Concepteur et d\'{e}veloppeur - Petites Annonces syst\`{e}me de gestion - Net Framework 3.0 C\# Nhibernate Windows form
	\item Concepteur de logiciels, d\'{e}veloppeur et l'analyse fonctionnelle - syst\`{e}me de facturation en ligne - PHP, CakePHP, Javascript jQuery.
	\item Concepteur de logiciels, d\'{e}veloppeur - syst\`{e}me \'{e}lectronique des factures (Bas\'{e} sur le syst\`{e}me de taxes locales) - Java, Groovy et Grails / jasper reports.
	\item R\'{e}f\'{e}rence de contact: Osvaldo Aufiero osvaldo@aufiero.com.ar
\end{rSubsection}


\begin{rSubsection}{MSA}{2006}{DBA Oracle 9i \& Postgres SQL}{Ciudad de Buenos Aires, Argentine}
	\item DBA de trois bases de donn\'{e}es de production li\'{e}s avec le syst\`{e}me de billetterie (sorte de Ticketek mais avec moins de stress)
	\item Mes taches \'{e}taient suivi de base de donn\'{e}es, de sauvegarde, de s\'{e}curit\'{e} SQL, les configurations de base de donn\'{e}es et optimisation de la requete pour les syst\`{e}mes productifs (Oracle9i) et qui constituent pour les projets \'{e}ventuels (Habituellement Postgres SQL)
	\item Parall\`{e}lement je ai eu quelques petits responsabilit\'{e}s dans des projets \'{e}ventuels en PHP et Python 
\end{rSubsection}


\begin{rSubsection}{Research for decision}{2003 - 2005}{D\'{e}veloppeur}{Ciudad de Buenos Aires, Argentine}
	\item Conception et d\'{e}veloppement de sondages dans eole / saxophone (langue et syst\`{e}me de sondage).
	\item La maintenance des serveurs, des machines et du r\'{e}seau
\end{rSubsection}

\begin{rSubsection}{Freelance}{2002 - 2009}{D\'{e}veloppeur, designer, architecte, DBA et la gestion de Client} {Ciudad de Buenos Aires, Argentine}
	\item Ce sont les projets que je ai d\'{e}velopp\'{e}s comme ind\'{e}pendants, plusieurs d'entre eux encore en usage, et je maintenue presque les un \`{a} deux ans.
	\item  Novembre 2002 to Juillet 2003 - pr\'{e}parations nutritionnelles syst\`{e}me de gestion pour un laboratoire parent\'{e}rale (UNANUT) - Visual Basic \& Microsoft Access (pris en charge jusqu'en 2005). R\'{e}f\'{e}rence de contact: Lorena Fazio faziolorena@yahoo.com.ar
	\item  Juin 2003 to Aout 2003 - syst\`{e}me de gestion des stocks pour une usine de boite - Visual Basic et Microsoft Access (pris en charge jusqu'en 2004). R\'{e}f\'{e}rence de contact: Gerardo Grimaldi grimaldi.gerardo@gmail.com
	\item Juillet 2009 to Mars 2010 - SFAP certification et des cours syst\`{e}me de gestion - PHP (http://www.facpce.org.ar/) (pris en charge jusqu'en 2011). R\'{e}f\'{e}rence de contact: Christian Milokanovich milo76@gmail.com
\end{rSubsection}



\end{rSection}


%----------------------------------------------------------------------------------------
%	OPEN SOURCE SECTION
%----------------------------------------------------------------------------------------
%------------------------------------------------
\begin{rSection}{Exp\'{e}rience en open source - projets communautaires}

\begin{rSubsection}{Projet personnel}{Mars - Mai 2014}{Making work Scala + Play + MongoDB}{Malaga}
	\item Rassembler une technologie existante
	\item Enti\`{e}rement d\'{e}velopp\'{e} avec Scala and Play. Il a quelques d\'{e}pendances java (Jackson pour JSon triage)
	\item https://github.com/sbragagnolo/mongodb
\end{rSubsection}

\begin{rSubsection}{Projet personnel}{June - August 2014}{Social secure plugin + Play + MongoDB}{Malaga}
	\item Social secure backend mise en œuvre pour l'ex\'{e}cution sur MongoDB
	\item Enti\`{e}rement d\'{e}velopp\'{e} avec Scala and Play.
	\item https://github.com/sbragagnolo/SocialSecurePlayMongo
\end{rSubsection}


\begin{rSubsection}{Google summer of code}{May 2012 - September 2012}{
Inf\'{e}rence de type dans les langages dynamiques}{Ciudad de Buenos Aires, Argentine}
	\item Conception, le d\'{e}veloppement, la planification et la recherche dans le sujet de l'inf\'{e}rence de type.
	\item Encadr\'{e}s par Nicolas Passerini (npasserini@gmail.com).
	\item La proposition est inscrit http://gsoc2012.esug.org/projects/type-inference
	\item De ce projet, je ai mis en place un syst\`{e}me d'inf\'{e}rence de type concret pour Pharo Smalltalk 1.4, un graphique de m\'{e}thodes pour etre ex\'{e}cut\'{e} en tant que r\'{e}ponse de l'analyse d'une expression donn\'{e}e, et un objet enregistreur orient\'{e}.
	\item Je ai \'{e}galement le travail de tous les blogs progr\`{e}s r\'{e}alis\'{e}s et \'{e}laborer des d\'{e}cisions de travail dans un blog. Ce projet a \'{e}t\'{e} pr\'{e}sent\'{e} dans le ESUG de conf\'{e}rence 2012 - Gand - (http://www.esug.org/wiki/pier/Conferences/2012)
	\item Site - http://concretetypeinference.blogspot.fr/. 
	\item Type inf\'{e}rence et l'analyse appel graphe (Kwisatz Haderach) - http://ss3.gemstone.com/ss/ConcreteTypeInference.html
	\item Paule le poulpe; Object oriented Logger - http://smalltalkhub.com/mc/sbragagnolo/PLP/main
\end{rSubsection}

 
\begin{rSubsection}{Projet financ\'{e} ESUG}{ Mai 2011 - Septembre 2011}{D\'{e}veloppeur in DBXTalk}{Ciudad de Buenos Aires, Argentine}
	\item Conception et mise en œuvre de la liaison automatique et scaffolding pour DBXTalk (ex SqueakDBX) et Glorp.
	\item Ce projet avec le portage de Glorp au Pharo (par Guillermo Polito) ont \'{e}t\'{e} pr\'{e}sent\'{e}es dans ESUG 2011 - Edimbourg (http://www.esug.org/wiki/pier/Conferences/2011)
	\item  R\'{e}f\'{e}rence de contact:  Guillermo Polito guillermopolito@gmail.com
\end{rSubsection}

\begin{rSubsection}{Projet financ\'{e} FACPCE }{ 2009 }{D\'{e}veloppeur, designer}{Ciudad de Buenos Aires, Argentine}	
	\item PHP Cornucopia. Est un fullstack framework d\'{e}j\`{a} obsol\`{e}te. Il pr\'{e}voit:
	\begin{itemize}
		\item Collections
		\item Simple ORM configurable par metadata 
		\item Simple dependency injector
		\item HTML reification 
		\item Javascript generation par metadata (Requirements, etc)
		\item Request comme objet
		\item Session comme objet
	\end{itemize}
	\item https://github.com/sbragagnolo/cornucopia
\end{rSubsection}


\begin{rSubsection}{Projet personnel}{ 2007 }{D\'{e}veloppeur, designer}{Ciudad de Buenos Aires, Argentine}
	\item C Objects est une Framework. Il d\'{e}finit plusieurs fonctions de programmation orient\'{e}e objet  C. Il pr\'{e}voit:
	\begin{itemize}
		\item Collections (Dictionary, List) 
		\item Strings
		\item Serveur, 
		\item Automatons
		\item Threads 
		\item Mutex et Conditional
		\item La gestion des erreurs
		\item Destion de la m\'{e}moire
	\end{itemize}
	\item https://github.com/sbragagnolo/c-objetos
	\item R\'{e}f\'{e}rence de contact: nicolassouto@gmail.com
\end{rSubsection}


\end{rSection}

%----------------------------------------------------------------------------------------
%	TEACHING SECTION
%----------------------------------------------------------------------------------------

\begin{rSection}{Teaching experience}


\begin{rSubsection}{Universidad Tecnologica Nacional (UTN)}{Mars 2007 - Juillet 2012}{Ad honorem assistant d'enseignement sur les techniques de programmation avanc\'{e}es}{Ciudad de Buenos Aires, Argentine}
\item Enseigner les concepts suivants, techniques et outils
\begin{itemize}
	\item La programmation orient\'{e}e objet
	\begin{itemize}
		\item Patterns
		\item Methodologies TDD/BDD, DDD, Agile
		\item Re-factors
		\item Meta-programming
	\end{itemize}
	\item Architectures de base
	\item Technologies
	\begin{itemize}
		\item Maven 
		\item IDEs - Eclipse, Idea, Netbeans
		\item JUnit, ScalaTest
		\item SVN \& GIT
	\end{itemize}
	\item Les langages dynamiques
	\begin{itemize}
		\item  Scala
		\item  Python
		\item  Smalltalk
		\item  Self
	\end{itemize}
	\item Concepts modernes appliqu\'{e}es
	\begin{itemize}
		\item Traits and Mixins
		\item  Lambdas / anonymous functions
	\end{itemize}
\end{itemize}
\item R\'{e}f\'{e}rence de contact: npasserini@gmail.com
\end{rSubsection}

\begin{rSubsection}{Universidad Tecnologica Nacional (UTN)}{Mars 2007 - Decembre 2011}{Ad honorem assistant d'enseignement au paradigmes de programmation}{Ciudad de Buenos Aires, Argentine}
\item Enseigner les concepts suivants, techniques et outils
\begin{itemize}
	\item Paradigme orient\'{e} objet
	\begin{itemize}
		\item Pharo Smalltalk
	\end{itemize}
	\item Paradigme orient\'{e} fonctionnelle
	\begin{itemize}
		\item GHC
		\item WinHugs
	\end{itemize}
	\item Paradigme orient\'{e} logique
	\begin{itemize}
		\item  Swi prolog
	\end{itemize}
\end{itemize}
\item R\'{e}f\'{e}rence de contact: carlombardi@gmail.com
\end{rSubsection}


\begin{rSubsection}{Fundacion Proydesa}{Mars 2006 - Novembre 2007}{Oracle DBA instructeur}{Ciudad de Buenos Aires, Argentine}
	\item Dans cette base je ai travaill\'{e} comme instructeur de trois des quatre modules de base d'Oracle pour l'administration de base de donn\'{e}es
	\item SQL (Module 1)
	\item Engine Architecture (Module 2)
	\item Tuning (Module 4)
	\item Je ai aussi particip\'{e} en tant qu'instructeur dans l'instruction de nouveaux instructeurs pour les modules 1 et 2.
\end{rSubsection}


\end{rSection}



%----------------------------------------------------------------------------------------
%	EDUCATION SECTION
%----------------------------------------------------------------------------------------

\begin{rSection}{Education}


{\bf Universidad Tecnologica Nacional (UTN) - Ciudad de Buenos Aires - Argentine} \hfill {\em Janvier 2004 - Abandonn\'{e} en 2011 parce que les probl\`{e}mes de la famille, ayant des papiers pour prouver sept semestres de carri\`{e}re de 10} \\ 
Software Engineer  \\

{\bf Ing. Otto Krause - Ciudad de Buenos Aires - Argentina} \hfill {\em December 2001} \\ 
Technicien dans le calcul \\
\end{rSection}

%----------------------------------------------------------------------------------------
%	EXAMPLE SECTION
%----------------------------------------------------------------------------------------

%\begin{rSection}{Section Name}

%Section content\ldots

%\end{rSection}

%----------------------------------------------------------------------------------------

\end{document}
